\documentclass{article} 
\usepackage[yyyymmdd]{datetime}
\usepackage[nottoc]{tocbibind}
\usepackage{xurl}
\usepackage{array}
\usepackage[inline,shortlabels]{enumitem}
\usepackage{hhline}
\usepackage{multirow}
\usepackage{geometry}
\usepackage[dvipsnames]{xcolor}
\usepackage{colortbl}
\usepackage[framemethod=TikZ]{mdframed}
\usepackage{amsfonts}
\usepackage{mathtools}
\usepackage{tikz}
\usepackage{amsthm}
\usepackage{amsmath}
\newtheoremstyle{problemstyle}{3pt}{3pt}{\normalfont}{}{\bfseries}{\normalfont\bfseries:}{.5em}{}
\theoremstyle{problemstyle}
\newmdtheoremenv[
  linewidth=1pt,
  linecolor=RoyalBlue,
  backgroundcolor=RoyalBlue!10,
  roundcorner=5pt,
  innertopmargin=6pt,
  innerbottommargin=6pt,
  innerleftmargin=6pt,
  innerrightmargin=6pt,
  nobreak=true
]{problem}{Problem}

% Example
\newmdtheoremenv[
  linewidth=1pt,
  linecolor=ForestGreen,
  backgroundcolor=ForestGreen!10,
  roundcorner=5pt,
  nobreak=true
]{example}{Example}

% Theorem
\newmdtheoremenv[
  linewidth=1pt,
  linecolor=BrickRed,
  backgroundcolor=BrickRed!10,
  roundcorner=5pt,
  nobreak=true
]{theorem}{Theorem}

% Remark
\newmdtheoremenv[
  linewidth=1pt,
  linecolor=Goldenrod,
  backgroundcolor=Goldenrod!10,
  roundcorner=5pt,
  nobreak=true
]{remark}{Remark}

\newmdtheoremenv[
  linewidth=1pt,
  linecolor=Goldenrod,
  backgroundcolor=Goldenrod!10,
  roundcorner=5pt,
  nobreak=true
  ]{definition}{Definition}

% Solution
\newenvironment{solution}{%
  \begin{mdframed}[linewidth=0.8pt,linecolor=Gray,backgroundcolor=Gray!5,roundcorner=5pt, nobreak=true]%
  \noindent\textbf{Solution.}%
}{%
\hfill $ \diamond $ 
  \end{mdframed}%
}

\usepackage{listings}
\usepackage[utf8]{inputenc}
\setlength{\parindent}{0pt} % Don't indent new paragraphs
\setlength{\headheight}{24pt} 
% Make header with name and date etc.
\usepackage{fancyhdr}
\lhead{Pedro D. Llerenas\\M\'etodos Num\'ericos I}
\rhead{\today\\Tarea II}
\thispagestyle{fancy}

\newcommand{\Z}{\mathbb Z}
\newcommand{\Q}{\mathbb Q}
\newcommand{\R}{\mathbb R}
\newcommand{\C}{\mathbb C}
\newcommand{\N}{\mathbb N}
\newcommand{\abs}[1]{\lvert #1 \rvert}
\DeclareMathOperator{\Var}{\mathbf{Var}}
\DeclareMathOperator{\E}{\mathbf{E}}
\newcommand{\A}{\mathbf A}
\newcommand{\x}{\mathbf x}
\newcommand{\bv}{\mathbf b}

\begin{document}
  \section{Problemas Lineales}\label{sec:problemas_lineales} % (fold)
  
  Queremos estudiar problemas del tipo
  \[
    \A \x = \bv,
  \]
  donde 
  \[
    \A = [a_{i j}]_{i, j = 1}^{n}
  \]
  \begin{itemize}
    \item Solucion de matriz diagonal con 
      \[
        a_{ii} \neq 0\qquad a_{ij} = 0 \quad \forall i\neq j
      \]
      Entonces,
      \[
        \A\x = \bv
      \]
      tiene soluciones $ x_i =b_{i}/a_{ii} $.
    \item Soluciones de matriz triangular inferior
      % \begin{cases*}
      %   a_{ij} = l_{ij} & i\leq j\\
      %   a_{ij} = 0 & i> j
      % \end{cases*}
  
      resolvemos de arriba hacia abajo:
      \begin{align*}
        x_1 &= b_{1}/l_{11},\\
        x_2 &= \frac{b_2 - l_{11}x_1}{l_{22}}\\
        x_k &= \frac{b_{k} - \sum l_{ii}x_i}{l_{kk}}
      \end{align*}
    \item Solcionces de matriz triangular superior
      % \begin{cases}
      %   a_{ij} = l_{ij} & i\geq j\\
      %   a_{ij} = 0 & i< j
      % \end{cases}


  \end{itemize}
  \subsection{Gauss-Jordan}
  Este algoritmo consiste en traingulizar la matriz. \textbf{INSERTAR FORMULA:}
  Como este m'etodo es costoso, hacemos \textit{pivoteo}, que consiste en cambiar columnas y filas, de tal manera que ponemos el mayor valor de la matriz en la diagonal, luego continuamos el proceso. Esto se hace para que las divisiones sean entre n\'umeros grandes, y tener un algoritmo consistente. 

  \subsection{Calcular inversas}
  Para encontrar la matriz inversa mediante Gauss-Jordan aumentado, estamos solucionando el sistema de ecuaciones $ n $ veces. Entonces, con los algoritmos vistos para resolver el sistema de ecuaciones, podemos hacerlo para $ \bv = e_i $, donde $ e_i = (0,\dots, 1, \dots, 0) $. 
  \subsection{Descomposicion LU}
  El m\'etodo de Grouth consiste en descomponer la matriz en un producto de matrices triangulares inferiores y superiores.
  \begin{align*}
    \begin{bmatrix}
      a_{11} & a_{12} & \dots & a_{1n}\\
      a_{21} & a_{22} &\dots & a_{2n}\\
      \vdots & & &\vdots\\
      a_{n1} & a_{n2} & \dots & a_{nn}
    \end{bmatrix} &=
    \begin{bmatrix}
      l_{11} & 0 & \dots & 0\\
      l_{21} & l_{22} &\dots & 0\\
      \vdots & & &\vdots\\
      l_{n1} & l_{n2} & \dots & l_{nn}
    \end{bmatrix}  
    \begin{bmatrix}
      1 & u_{12} & \dots & u_{1n}\\
      0 & 1&\dots & u_{2n}\\
      \vdots & & &\vdots\\
      0 & 0 & \dots & 1 
    \end{bmatrix}  \\
    (n = 3)&=
    \begin{bmatrix}
      l_{11} & l_{11} u_{12} & l_{11}u_{12}\\
      l_{21} & l_{21}u_{12} + l_{22} & l_{21}a_{13} + l_{22}a_{23}\\
      l_{31} & l_{31}u_{12} + l_{32} & l_{31}u_{13}+ l_{32}u_{23}+l_{33}
    \end{bmatrix}
  \end{align*}
  Podemos calcular de manera expl\'icita:
\begin{align*}
  a_{11} &= l_{11}\\
  a_{21} &= l_{21}\\
  a_{12} &= l_{11}u_{12} \implies u_{12} = \displaystyle\frac{a_{12}}{l_{11}}
  a_{ik} &= Pending
\end{align*}

Tenemos

\begin{align*}
  l_{ij} &= a_{ij} - \sum_{k = 1}^{j-1}l_{ik}u_{kj}\\
  u_{ij}&= a_{ij} - \sum_{k=1}^{j-1}l_{ik}u_{kj}\\
  l_{ii} &= a_{ii} - \sum_{k=1}^{n-1}l_{ik}u_{ki}
\end{align*}

\subsection{Descomposicion de Cholevsky}
Consiste en descomponer la matriz Hermitian $ \A $ de forma $ LL^T $. Notemos que para la resolucio\'on de sistemas de ecuaciones, $ \A $ debe ser definida positiva.

  \begin{align*}
    \begin{bmatrix}
      a_{11} & a_{12} & \dots & a_{1n}\\
      a_{21} & a_{22} &\dots & a_{2n}\\
      \vdots & & &\vdots\\
      a_{n1} & a_{n2} & \dots & a_{nn}
    \end{bmatrix} &=
    \begin{bmatrix}
      l_{11} & 0 & \dots & 0\\
      l_{21} & l_{22} &\dots & 0\\
      \vdots & & &\vdots\\
      l_{n1} & l_{n2} & \dots & l_{nn}
    \end{bmatrix}  
    \begin{bmatrix}
      l_{11} & l_{12} & \dots & l_{1n}\\
      0 & l_{22}&\dots & l_{2n}\\
      \vdots & & &\vdots\\
      0 & 0 & \dots & l_{nn} 
    \end{bmatrix}  \\
    (n = 3)&=
    \begin{bmatrix}
      l^2_{11} & l_{11} l_{21} & l_{11}u_{31}\\
               &  l_{11}^2 + l_{22}^2& l_{21}l_{31} + l_{22}l_{12} \\
       & & l_{11}^2+l_{22}^2+l_{33}^2
    \end{bmatrix}
  \end{align*}

  \begin{align*}
    l_{ij} = a_{ij} - \sum_{k=1}^{j-1} l_{ik}l_{jk}
    l_{ii} = \sqrt{a_{ii} - \sum_{k=1}^{j-1}l_{ik}^2}
  \end{align*}

  % section Problemas Lineales (end)
  \section{Ecuacion de calor}\label{sec:ecuacion_de_calor} % (fold)
  \subsection{Ecuacion de calor en 1D}
  Definida por $ q = -K \frac{dQ}{dx} $, esta ecuaci\'on describe el flujo de calor en una dimensi\'on. Es decir, si aplicamos una fuente de calor a una vara, esta nos dice la temperatura en cada punto de la vara.
  
  \[
    K \frac{\partial \phi}{\partial x^2} + Q = 0.
  \]
  La derivada puede ser calculada de manera discreta:

  \[
    \frac{d\phi}{dx} = \frac{\phi_i-\phi_{i+1}}{\Delta x}
  \]
  \begin{align*}
    \frac{d^2\phi}{dx^2} = \frac{\phi_{i-1}-2\phi_{i}+\phi_{i+1}}{\Delta x^2}
  \end{align*}

  
  \begin{align*}
    K\frac{\phi_{i-1}-2\phi_{i}+\phi_{i+1}}{\Delta x} +Q\Delta x = 0.
  \end{align*}
  Es decir, 
  \begin{align*}
    \frac{K}{dx^2}\begin{pmatrix}
      -1 & 2 & -1
    \end{pmatrix}\begin{bmatrix}
    \phi_{i-1}\\
    \phi_{i}\\
    \phi_{i+1}
    \end{bmatrix} = Q.
  \end{align*}
   \begin{align*}
    \begin{bmatrix}
      2 & -1 & & & &\\
      -1 & 2 & -1 & & &\\
    \end{bmatrix}
   \end{align*}
  
  % section Ecuacion de calor (end)

  \section{Metodo de Jacobi}\label{sec:metodo_de_jacobi} % (fold)
  Queremos resolver $ N^{-1}(Ax - b) = 0 $. Comenzamos por el caso $ N = D $, la diagonal de $ A $. Entonces, descomponemos $ A = L + D + U $, y tenemos
  \begin{align*}
    D^{-1} (L+D+U x - b) &= 0\\
    (D^{-1} (L+U) x^{n-1} + x^{n} - D^{-1}b) = 0\\
    \implies A_{ii} x^n = b_i - \sum_{j \neq i} A_{ij}x_j^{n-1}
  \end{align*}
  % section Metodo de Jacobi (end)
  
\end{document}
