\documentclass{article}
\usepackage[T1]{fontenc}
\usepackage[utf8]{inputenc}
\usepackage[yyyymmdd]{datetime}
\usepackage[nottoc]{tocbibind}
\usepackage{pgfplots}
\pgfplotsset{compat=1.18}
\usepackage{url}
\usepackage{array}
\usepackage{bm}
\usepackage{float}
\usepackage[inline,shortlabels]{enumitem}
\usepackage{hhline}
\usepackage{multirow}
\usepackage{geometry}
\usepackage[dvipsnames]{xcolor}
\usepackage{colortbl}
\usepackage[framemethod=TikZ]{mdframed}
\usepackage{amsfonts}
\usepackage{amsmath}
\usepackage{tikz}
\usepackage{amsthm}
\newtheoremstyle{problemstyle}{3pt}{3pt}{\normalfont}{}{\bfseries}{\normalfont\bfseries:}{.5em}{}
\theoremstyle{problemstyle}
\newmdtheoremenv[
  linewidth=1pt,
  linecolor=RoyalBlue,
  backgroundcolor=RoyalBlue!10,
  roundcorner=5pt,
  innertopmargin=6pt,
  innerbottommargin=6pt,
  innerleftmargin=6pt,
  innerrightmargin=6pt,
  nobreak=true
]{problem}{Problem}

% Example
\newmdtheoremenv[
  linewidth=1pt,
  linecolor=ForestGreen,
  backgroundcolor=ForestGreen!10,
  roundcorner=5pt,
  nobreak=true
]{example}{Example}

% Theorem
\newmdtheoremenv[
  linewidth=1pt,
  linecolor=BrickRed,
  backgroundcolor=BrickRed!10,
  roundcorner=5pt,
  nobreak=true
]{theorem}{Theorem}

% Remark
\newmdtheoremenv[
  linewidth=1pt,
  linecolor=Goldenrod,
  backgroundcolor=Goldenrod!10,
  roundcorner=5pt,
  nobreak=true
]{remark}{Remark}

% Solution
\newenvironment{solution}{%
  \begin{mdframed}[linewidth=0.8pt,linecolor=Gray,backgroundcolor=Gray!5,roundcorner=5pt]%
  \noindent\textbf{Solution.}%
}{%
\hfill $ \diamond $ 
  \end{mdframed}%
}

\usepackage{listings}

\lstset{breaklines=true}
\definecolor{vscode-blue}{RGB}{64, 116, 160}   % Keywords
\definecolor{vscode-green}{RGB}{95, 203, 114}   % Strings
\definecolor{vscode-gray}{RGB}{128, 128, 128}   % Comments
\definecolor{vscode-orange}{RGB}{206, 145, 120} % Numbers/Constants
\definecolor{vscode-purple}{RGB}{204, 124, 255} % Functions/Types
\definecolor{vscode-cyan}{RGB}{155, 199, 217}   % Preprocessor

\lstdefinestyle{visual-studio}{
    language=C,
    backgroundcolor=\color{Gray!10},
    commentstyle=\color{vscode-gray}\ttfamily,
    keywordstyle=\color{vscode-blue}\ttfamily,
    stringstyle=\color{vscode-green}\ttfamily,
    numberstyle=\color{vscode-orange}\ttfamily,
    identifierstyle=\color{black}\ttfamily,
    breaklines=true,
    showstringspaces=false,
    tabsize=4,
    % Additional settings for more Visual Studio feel
    emph=[1]{void, int, double, float, char, bool, short, long, signed,
    unsigned, class, struct, enum, union, typedef, template}, % Common types
    emphstyle=[1]\color{vscode-purple},
    emph=[2]{printf, scanf, cin, cout, return, new, delete, if, else, while,
    do, for, switch, case, default, break, continue, goto, throw, try, catch,
  const, static, extern, volatile, register, restrict, inline, explicit,
virtual, friend, namespace, using, private, protected, public, operator}, %
Standard keywords
    emphstyle=[2]\color{vscode-blue},
            keywordstyle=[2]\color{vscode-cyan}\ttfamily\bfseries,
    morekeywords={\#include, \#define, \#ifdef, \#ifndef, \#endif, \#pragma},
    basicstyle=\small
}

% Make header with name and date etc.
\usepackage{fancyhdr}
\lhead{Pedro D. Llerenas\\M\'etodos Num\'ericos I}
\rhead{\today\\Tarea V}
\thispagestyle{fancy}

\setlength{\parindent}{0pt} % Don't indent new paragraphs
\setlength{\headheight}{24pt} 

\newcommand{\Z}{\mathbb Z}
\newcommand{\Q}{\mathbb Q}
\newcommand{\R}{\mathbb R}
\newcommand{\C}{\mathbb C}
\newcommand{\N}{\mathbb N}

\begin{document}

%begin problem 1
\begin{problem}
Desarrollar un programa que calcule los $ k $ valores propios m\'as grandes en magnitud de
una matriz $ A $ mediante el m\'etodo de la potencia. \textbf{(5 puntos)}
\end{problem}
%end problem 1

%begin solution 1
\begin{solution}
	Este algoritmo consiste de dos pedazos funciones principales,
	\texttt{iterative\_power} y \texttt{get\_m\_largest\_eigenvalues}. El primer
	m\'etodo tiene la funci\'on de encontrar el eigenvalor de mayor magnitud. La
	segunda funci\'on utiliza la primera de manera iterativa. Despu\'es de
	encontrar el m\'as grande, utilizamos una t\'ecnica de \textit{deflaci\'on},
	que consiste en aplicar el mismo m\'etodo de la potencia, pero con una matriz
	modificada a la original. Espec\'ificamente, si queremos el eigenvalor $ m $
	m\'as grande, los eigenvalores de la nueva matriz son exactamente los mismos,
	excepto los $ m-1 $ m\'as grandes, que son reemplazados por el eigenvalor 0.
	Esto asegura que la aplicaci\'on del m\'etodo de la potencia encuentre el
	eigenvalor especificado.


	\textbf{Ejemplos:}
	\begin{enumerate}
    \item \textbf{Eigen\_3x3.txt}
		      \begin{align*}
			      \begin{bmatrix}
				      5.0    & -1.778 & 0.0    \\
				      -1.778 & 9.0    & -1.778 \\
				      0.0    & -1.778 & 10.0
			      \end{bmatrix}
		      \end{align*}
		      Como vector inicial, usaremos $ \mathbf{x}_0 = \begin{bmatrix}
				      1 & 1 & 1
			      \end{bmatrix}^{T} $
		      Ejecutando nuestro programa con
		      \begin{center}
			      \texttt{make run-p1
				      ARGS="t5a/Eigen\_3x3.txt t5a/x0\_3.txt 3 1e-6 500"},
		      \end{center}
		      obtenemos los 3 eigenvalores m\'as grandes (en este caso, todos).
		      \begin{table}[H]
			      \begin{center}
				      \begin{tabular}{|l|l|}
					      \hline
					      Eigenvector           & Eigenvalue \\
					      \hline
					      \rule{0pt}{1.5em}
					      $ \begin{bmatrix}
							        0.18 & -0.64 & 0.74
						        \end{bmatrix} $  & 11.538403     \\
					      [0.5em]
					      \hline
					      \rule{0pt}{1.5em}
					      $ \begin{bmatrix}
							        -0.37 & 0.92 & -0.17
						        \end{bmatrix} $ & 8.213811       \\
					      [0.5em]
					      \hline
					      \rule{0pt}{1.5em}
					      $ \begin{bmatrix}
							        0.91 & -0.42 & 0.02
						        \end{bmatrix} $  & 4.247786      \\
					      [0.5em]
					      \hline
				      \end{tabular}
			      \end{center}
			      \caption{Eigenvalores de Eigen\_3x3.txt}\label{tab:3x3}
		      \end{table}
		      Como verificaci\'on, utilizamos la librer\'ia de \texttt{numpy}. Corremos el programa (antes activando el virtual environment, e instalando numpy mediante \texttt{pip install numpy} si es necesario) mediante
		      \begin{center}
			      \texttt{python3 p1\textunderscore verify.py t5a/Eigen\textunderscore3x3.txt 3 1}
		      \end{center}
		      Este nos genera los valores
		      \begin{table}[H]
			      \begin{center}
				      \begin{tabular}{|l|l|}
					      \hline
					      Eigenvector           & Eigenvalue \\
					      \hline
					      \rule{0pt}{1.5em}
					      $ \begin{bmatrix}
							        0.18 & -0.64 & 0.74
						        \end{bmatrix} $  & 11.538405     \\
					      [0.5em]
					      \hline
					      \rule{0pt}{1.5em}
					      $ \begin{bmatrix}
							        -0.37 & 0.92 & -0.17
						        \end{bmatrix} $ & 8.213809       \\
					      [0.5em]
					      \hline
					      \rule{0pt}{1.5em}
					      $ \begin{bmatrix}
							        0.91 & -0.42 & 0.02
						        \end{bmatrix} $  & 4.247786      \\
					      [0.5em]
					      \hline
				      \end{tabular}
			      \end{center}
			      \caption{Eigenvalores de Eigen\_3x3.txt verificados con numpy.}\label{tab:py3x3}
		      \end{table}
		      Podemos observar muy poca diferencia entre los eigenvalores, por lo
		      que podemos asumir que han sido verificados como correctos.

        \item \textbf{Eigen\_5x5.txt}
		      \begin{align*}
			      \begin{bmatrix}
				      0.2 & 0.1 & 1  & 1 & 0   \\
				      0.1 & 4   & -1 & 1 & -1  \\
				      1   & -1  & 60 & 0 & -2  \\
				      1   & 1   & 0  & 8 & 4   \\
				      0   & -1  & -2 & 4 & 700
			      \end{bmatrix}
		      \end{align*}
		      Como vector inicial, usaremos $ \mathbf{x}_0 = \begin{bmatrix}
				      1 & 1 & 1 & 1 & 1
			      \end{bmatrix}^{T} $
		      Ejecutando nuestro programa con
		      \begin{center}
			      \texttt{make run-p1
				      ARGS="t5a/Eigen\_5x5.txt t5a/x0\_5.txt 3 1e-6 500"},
		      \end{center}
		      obtenemos los 3 eigenvalores m\'as grandes (en este caso, todos).
		      \begin{table}[H]
			      \begin{center}
				      \begin{tabular}{|l|l|}
					      \hline
					      $ n $ & Eigenvalue \\
					      \hline
					      1     & 700.030781 \\
					      \hline
					      2     & 60.028366  \\
					      \hline
					      3     & 8.338448   \\
					      \hline
				      \end{tabular}
			      \end{center}
			      \caption{Eigenvalores de Eigen\_5x5.txt}\label{tab:5x5}
		      \end{table}
		      Como verificaci\'on, utilizamos la librer\'ia de \texttt{numpy}. Corremos el programa (antes activando el virtual environment, e instalando numpy mediante \texttt{pip install numpy} si es necesario) mediante
		      \begin{center}
			      \texttt{python3 p1\textunderscore verify.py t5a/Eigen\textunderscore5x5.txt 3 1}
		      \end{center}
		      Este nos genera los valores
		      \begin{table}[H]
			      \begin{center}
				      \begin{tabular}{|l|l|}
					      \hline
					      n & Eigenvalue \\
					      \hline
					      1 & 700.030781 \\
					      \hline
					      2 & 60.028366  \\
					      \hline
					      3 & 8.338447   \\
					      \hline
				      \end{tabular}
			      \end{center}
			      \caption{Eigenvalores de Eigen\_5x5.txt verificados con numpy.}\label{tab:py5x5}
		      \end{table}
		      Podemos observar muy poca diferencia entre los eigenvalores, por lo
		      que podemos asumir que han sido verificados como correctos.
        \item \textbf{Eigen\_50x50.txt}
		      \begin{align*}
			      \begin{bmatrix}
				      10.000 & 0.084  & 0.039  & \cdots & 0.052   & 0.078   & 0.007   \\
				      0.084  & 20.000 & 0.095  & \cdots & 0.040   & 0.053   & 0.040   \\
				      0.039  & 0.095  & 30.000 & \cdots & 0.084   & 0.013   & 0.052   \\
				      \vdots & \vdots & \vdots & \ddots & \vdots  & \vdots  & \vdots  \\
				      0.081  & 0.044  & 0.095  & \cdots & 380.000 & 0.038   & 0.081   \\
				      0.092  & 0.092  & 0.005  & \cdots & 0.041   & 490.000 & 0.092   \\
				      0.007  & 0.040  & 0.052  & \cdots & 0.078   & 0.078   & 500.000 \\
			      \end{bmatrix}
		      \end{align*}
		      Como vector inicial, usaremos $ \mathbf{x}_0 = \begin{bmatrix}
				      1 & 1 & \cdots & 1 & 1
			      \end{bmatrix}^{T} $
		      Ejecutando nuestro programa con
		      \begin{center}
			      \texttt{make run-p1
				      ARGS="t5a/Eigen\_50x50.txt t5a/x0\_50.txt 7 1e-6 500"},
		      \end{center}
		      obtenemos los73 eigenvalores m\'as grandes (en este caso, todos).
		      \begin{table}[H]
			      \begin{center}
				      \begin{tabular}{|l|l|}
					      \hline
					      $ n $ & Eigenvalue \\
					      \hline
					      1     & 500.001591 \\
					      \hline
					      2     & 490.001430 \\
					      \hline
					      3     & 480.000520 \\
					      \hline
					      4     & 470.001067 \\
					      \hline
					      5     & 460.001265 \\
					      \hline
					      6     & 450.000481 \\
					      \hline
					      7     & 440.000418 \\
					      \hline
				      \end{tabular}
			      \end{center}
			      \caption{Eigenvalores de Eigen\_50x50.txt}\label{tab:50x50}
		      \end{table}
		      Como verificaci\'on, utilizamos la librer\'ia de \texttt{numpy}. Corremos el programa (antes activando el virtual environment, e instalando numpy mediante \texttt{pip install numpy} si es necesario) mediante
		      \begin{center}
			      \texttt{python3 p1\textunderscore verify.py t5a/Eigen\textunderscore50x50.txt 7 1}
		      \end{center}
		      Este nos genera los valores
		      \begin{table}[H]
			      \begin{center}
				      \begin{tabular}{|l|l|}
					      \hline
					      n & Eigenvalue \\
					      \hline
					      1 & 500.001543 \\
					      \hline
					      2 & 490.001431 \\
					      \hline
					      3 & 480.000527 \\
					      \hline
					      4 & 470.001058 \\
					      \hline
					      5 & 460.001269 \\
					      \hline
					      6 & 450.000483 \\
					      \hline
					      7 & 440.000419 \\
					      \hline
				      \end{tabular}
			      \end{center}
			      \caption{Eigenvalores de Eigen\_50x50.txt verificados con numpy.}\label{tab:py50x50}
		      \end{table}
		      Podemos observar muy poca diferencia entre los eigenvalores, por lo
		      que podemos asumir que han sido verificados como correctos.

        \item \textbf{Eigen\_125x125.txt}
		      \begin{align*}
			      \begin{bmatrix}
				      9.566e-05 & 0         & 0      & \cdots & 0         & 0         & 0         \\
				      0         & 1.965e-04 & 0      & \cdots & 0         & 0         & 0         \\
				      0 & 0 & 1      & \cdots & 0         & 0         & 0         \\
				      \vdots    & \vdots    & \vdots & \ddots & \vdots    & \vdots    & \vdots    \\
				      0         & 0         & 0      & \cdots & 3.802e-04 & 1.356e-06 & 0         \\
				      0         & 0         & 0      & \cdots & 1.356e-06 & 3.701e-04 & 0         \\
				      0         & 0         & 0      & \cdots & 0         & 0         & 3.522e-04 \\
			      \end{bmatrix}
		      \end{align*}
		      Como vector inicial, usaremos $ \mathbf{x}_0 = \begin{bmatrix}
				      1 & 1 & 1 & 1 & 1
			      \end{bmatrix}^{T} $
		      Ejecutando nuestro programa con
		      \begin{center}
			      \texttt{make run-p1
				      ARGS="t5a/Eigen\_125x125.txt t5a/x0\_125.txt 7 1e-6 500"},
		      \end{center}
		      obtenemos los 7 eigenvalores m\'as grandes.
		      \begin{table}[H]
			      \begin{center}
				      \begin{tabular}{|l|l|}
					      \hline
					      $ n $ & Eigenvalue \\
					      \hline
					      1     &1.000000\\
					      \hline
					      2     & 0.000598\\
					      \hline
					      3     & 0.000581\\
					      \hline
					      4     & 0.000561\\
					      \hline
					      5     & 0.000556\\
					      \hline
					      6     & 0.000551\\
					      \hline
					      7     & 0.000548\\
					      \hline
				      \end{tabular}
			      \end{center}
			      \caption{Eigenvalores m\'as grandes de Eigen\_125x125.txt}\label{tab:125x125}
		      \end{table}
		      Como verificaci\'on, utilizamos la librer\'ia de \texttt{numpy}. Corremos el programa (antes activando el virtual environment, e instalando numpy mediante \texttt{pip install numpy} si es necesario) mediante
		      \begin{center}
			      \texttt{python3 p1\textunderscore verify.py t5a/Eigen\textunderscore125x125.txt 18 1}
		      \end{center}
		      Este nos genera los valores
		      \begin{table}[H]
			      \begin{center}
				      \begin{tabular}{|l|l|}
					      \hline
					      n & Eigenvalue \\
					      \hline
					      1 & 1.000000\\
					      \hline
					      2 & 1.000000\\
					      \hline
                \vdots & \vdots\\
					      \hline
					      12 & 1.000000\\
					      \hline
					      13 & 0.000598\\
					      \hline
					      14 & 0.000581\\
					      \hline
					      15 & 0.000561\\
					      \hline
					      16 & 0.000556\\
					      \hline
					      17 & 0.000551\\
					      \hline
					      18 & 0.000548\\
					      \hline
				      \end{tabular}
			      \end{center}
			      \caption{Eigenvalores m\'as grandes de Eigen\_125x125.txt verificados con numpy.}\label{tab:py125x125}
		      \end{table}
          Por la naturaleza del algoritmo, no se va a encontrar el mismo
          eigenvalor en la siguiente iteraci\'on. Entonces, encontramos los 7
          eigenvalores m\'as grandes no repetidos. Como podemos confirmar,
          estos corresponden a los que regresa numpy.
	\end{enumerate}
\end{solution}
%end solution 1

%begin problem 2
\begin{problem}
Desarrollar un programa que calcule los $ k $ valores propios m\'as peque\~nos en magnitud de
una matriz $ A $ mediante el m\'etodo de la potencia inversa. \textbf{(5 puntos)}
\end{problem}
%end problem 2
%begin solution 1
\begin{solution}
  Para encontrar los eigenvalores m\'as peque\~nos, realizamos el m\'etodo de
  la potencia inversa. Despu\'es de cada iteraci\'on, realizamos un ajuste en
  los vectores para ortogonalizarlos con los eigenvalores anteriores,
  asegurando que el nuevo eigenvalor se encuentre en otro eigenespacio,
  permitiendo encontrar los otros eigenvalores m\'as peque\~nos.
	\begin{enumerate}
    \item \textbf{Eigen\_3x3.txt}
		      \begin{align*}
			      \begin{bmatrix}
				      5.0    & -1.778 & 0.0    \\
				      -1.778 & 9.0    & -1.778 \\
				      0.0    & -1.778 & 10.0
			      \end{bmatrix}
		      \end{align*}
		      Como vector inicial, usaremos $ \mathbf{x}_0 = \begin{bmatrix}
				      1 & 1 & 1
			      \end{bmatrix}^{T} $
		      Ejecutando nuestro programa con
		      \begin{center}
			      \texttt{make run-p2
				      ARGS="t5a/Eigen\_3x3.txt t5a/x0\_3.txt 3 1e-6 500"},
		      \end{center}
		      obtenemos los 3 eigenvalores m\'as peque\~nos (en este caso, todos).
		      \begin{table}[H]
			      \begin{center}
				      \begin{tabular}{|l|l|}
					      \hline
					      Eigenvector           & Eigenvalue \\
					      \hline
					      \rule{0pt}{1.5em}
					      $ \begin{bmatrix}
							        0.91 & -0.42 & 0.02
						        \end{bmatrix} $  & 4.247786      \\
					      [0.5em]
					      \hline
					      \rule{0pt}{1.5em}
					      $ \begin{bmatrix}
							        -0.37 & 0.92 & -0.17
						        \end{bmatrix} $ & 8.213811       \\
					      [0.5em]
					      \hline
					      \rule{0pt}{1.5em}
					      $ \begin{bmatrix}
							        0.18 & -0.64 & 0.74
						        \end{bmatrix} $  & 11.538403     \\
					      [0.5em]
					      \hline
				      \end{tabular}
			      \end{center}
			      \caption{3 eigenvalores m\'as peque\~nos de Eigen\_3x3.txt}\label{tab:inv3x3}
		      \end{table}
		      Como verificaci\'on, utilizamos la librer\'ia de \texttt{numpy}. Corremos el programa (antes activando el virtual environment, e instalando numpy mediante \texttt{pip install numpy} si es necesario) mediante
		      \begin{center}
			      \texttt{python3 p1\textunderscore verify.py t5a/Eigen\textunderscore3x3.txt 3 0}
		      \end{center}
		      Este nos genera los valores
		      \begin{table}[H]
			      \begin{center}
				      \begin{tabular}{|l|l|}
					      \hline
					      Eigenvector           & Eigenvalue \\
					      \hline
					      \rule{0pt}{1.5em}
					      $ \begin{bmatrix}
							        0.91 & -0.42 & 0.02
						        \end{bmatrix} $  & 4.247786      \\
					      [0.5em]
					      \hline
					      \rule{0pt}{1.5em}
					      $ \begin{bmatrix}
							        -0.37 & 0.92 & -0.17
						        \end{bmatrix} $ & 8.213809       \\
					      [0.5em]
					      \hline
					      \rule{0pt}{1.5em}
					      $ \begin{bmatrix}
							        0.18 & -0.64 & 0.74
						        \end{bmatrix} $  & 11.538405     \\
					      [0.5em]
					      \hline
				      \end{tabular}
			      \end{center}
			      \caption{3 eigenvalores m\'as peque\~nos de Eigen\_3x3.txt verificados con numpy.}\label{tab:invpy3x3}
		      \end{table}
		      Podemos observar muy poca diferencia entre los eigenvalores, por lo
		      que podemos asumir que han sido verificados como correctos.

        \item \textbf{Eigen\_5x5.txt}
		      \begin{align*}
			      \begin{bmatrix}
				      0.2 & 0.1 & 1  & 1 & 0   \\
				      0.1 & 4   & -1 & 1 & -1  \\
				      1   & -1  & 60 & 0 & -2  \\
				      1   & 1   & 0  & 8 & 4   \\
				      0   & -1  & -2 & 4 & 700
			      \end{bmatrix}
		      \end{align*}
		      Como vector inicial, usaremos $ \mathbf{x}_0 = \begin{bmatrix}
				      1 & 1 & 1 & 1 & 1
			      \end{bmatrix}^{T} $
		      Ejecutando nuestro programa con
		      \begin{center}
			      \texttt{make run-p2
				      ARGS="t5a/Eigen\_5x5.txt t5a/x0\_5.txt 3 1e-6 500"},
		      \end{center}
		      obtenemos los 3 eigenvalores m\'as peque\~nos.
		      \begin{table}[H]
			      \begin{center}
				      \begin{tabular}{|l|l|}
                \hline
					      n & Eigenvalue \\
					      \hline
					      1 & 0.057075 \\
					      \hline
					      2 & 3.745331\\
					      \hline
					      3 & 8.338447   \\
					      \hline
				      \end{tabular}
			      \end{center}
			      \caption{Eigenvalores de Eigen\_5x5.txt}\label{tab:inv5x5}
		      \end{table}
		      Como verificaci\'on, utilizamos la librer\'ia de \texttt{numpy}. Corremos el programa (antes activando el virtual environment, e instalando numpy mediante \texttt{pip install numpy} si es necesario) mediante
		      \begin{center}
			      \texttt{python3 p1\textunderscore verify.py t5a/Eigen\textunderscore5x5.txt 3 0}
		      \end{center}
		      Este nos genera los valores
		      \begin{table}[H]
			      \begin{center}
				      \begin{tabular}{|l|l|}
					      \hline
					      n & Eigenvalue \\
					      \hline
					      1 & 0.057075 \\
					      \hline
					      2 & 3.745331\\
					      \hline
					      3 & 8.338447   \\
					      \hline
				      \end{tabular}
			      \end{center}
			      \caption{3 eigenvalores m\'as peque\~nos de Eigen\_5x5.txt verificados con numpy.}\label{tab:invpy5x5}
		      \end{table}
		      Podemos observar muy poca diferencia entre los eigenvalores, por lo
		      que podemos asumir que han sido verificados como correctos.
        \item \textbf{Eigen\_50x50.txt}
		      \begin{align*}
			      \begin{bmatrix}
				      10.000 & 0.084  & 0.039  & \cdots & 0.052   & 0.078   & 0.007   \\
				      0.084  & 20.000 & 0.095  & \cdots & 0.040   & 0.053   & 0.040   \\
				      0.039  & 0.095  & 30.000 & \cdots & 0.084   & 0.013   & 0.052   \\
				      \vdots & \vdots & \vdots & \ddots & \vdots  & \vdots  & \vdots  \\
				      0.081  & 0.044  & 0.095  & \cdots & 380.000 & 0.038   & 0.081   \\
				      0.092  & 0.092  & 0.005  & \cdots & 0.041   & 490.000 & 0.092   \\
				      0.007  & 0.040  & 0.052  & \cdots & 0.078   & 0.078   & 500.000 \\
			      \end{bmatrix}
		      \end{align*}
		      Como vector inicial, usaremos $ \mathbf{x}_0 = \begin{bmatrix}
				      1 & 1 & 1 & 1 & 1
			      \end{bmatrix}^{T} $
		      Ejecutando nuestro programa con
		      \begin{center}
			      \texttt{make run-p2
				      ARGS="t5a/Eigen\_50x50.txt t5a/x0\_50.txt 7 1e-6 500"},
		      \end{center}
		      obtenemos los 7 eigenvalores m\'as peque\~nos.
		      \begin{table}[H]
			      \begin{center}
				      \begin{tabular}{|l|l|}
					      \hline
					      $ n $ & Eigenvalue \\
					      \hline
					      1     & 9.998050\\
					      \hline
					      2     & 19.998794\\
					      \hline
					      3     & 29.998932\\
					      \hline
					      4     & 39.999764\\
					      \hline
					      5     & 49.998366\\
					      \hline
					      6     & 59.999846\\
					      \hline
					      7     & 69.999136\\
					      \hline
				      \end{tabular}
			      \end{center}
			      \caption{7 eigenvalores m\'as peque\~nos de Eigen\_50x50.txt}\label{tab:inv50x50}
		      \end{table}
		      Como verificaci\'on, utilizamos la librer\'ia de \texttt{numpy}. Corremos el programa (antes activando el virtual environment, e instalando numpy mediante \texttt{pip install numpy} si es necesario) mediante
		      \begin{center}
			      \texttt{python3 p1\textunderscore verify.py t5a/Eigen\textunderscore50x50.txt 7 0}
		      \end{center}
		      Este nos genera los valores
		      \begin{table}[H]
			      \begin{center}
				      \begin{tabular}{|l|l|}
					      \hline
					      n & Eigenvalue \\
					      \hline
					      1 & 9.998050\\
					      \hline
					      2 & 19.998794\\
					      \hline
					      3 & 29.998931\\
					      \hline
					      4 & 39.999763\\
					      \hline
					      5 & 49.998366\\
					      \hline
					      6 & 59.999845\\
					      \hline
					      7 & 69.999135\\
					      \hline
				      \end{tabular}
			      \end{center}
			      \caption{Eigenvalores de Eigen\_50x50.txt verificados con numpy.}\label{tab:invpy50x50}
		      \end{table}
		      Podemos observar muy poca diferencia entre los eigenvalores, por lo
		      que podemos asumir que han sido verificados como correctos.

        \item \textbf{Eigen\_125x125.txt}
		      \begin{align*}
			      \begin{bmatrix}
				      9.566e-05 & 0         & 0      & \cdots & 0         & 0         & 0         \\
				      0         & 1.965e-04 & 0      & \cdots & 0         & 0         & 0         \\
				      0 & 0 & 1      & \cdots & 0         & 0         & 0         \\
				      \vdots    & \vdots    & \vdots & \ddots & \vdots    & \vdots    & \vdots    \\
				      0         & 0         & 0      & \cdots & 3.802e-04 & 1.356e-06 & 0         \\
				      0         & 0         & 0      & \cdots & 1.356e-06 & 3.701e-04 & 0         \\
				      0         & 0         & 0      & \cdots & 0         & 0         & 3.522e-04 \\
			      \end{bmatrix}
		      \end{align*}
		      Como vector inicial, usaremos $ \mathbf{x}_0 = \begin{bmatrix}
				      1 & 1 & \cdots & 1 & 1
			      \end{bmatrix}^{T} $
		      Ejecutando nuestro programa con
		      \begin{center}
			      \texttt{make run-p2
				      ARGS="t5a/Eigen\_125x125.txt t5a/x0\_125.txt 7 1e-9 500"},
		      \end{center}
		      obtenemos los 7 eigenvalores m\'as peque\~nos.
		      \begin{table}[H]
			      \begin{center}
				      \begin{tabular}{|l|l|}
					      \hline
					      $ n $ & Eigenvalue \\
					      \hline
					      1     & 0.000002\\
					      \hline
					      2     & 0.000008\\
					      \hline
					      3     & 0.000017\\
					      \hline
					      4     & 0.000018\\
					      \hline
					      5     & 0.000026\\
					      \hline
					      6     & 0.000036\\
					      \hline
					      7     & 0.000038\\
					      \hline
				      \end{tabular}
			      \end{center}
			      \caption{Eigenvalores de Eigen\_125x125.txt}\label{tab:inv125x125}
		      \end{table}
		      Como verificaci\'on, utilizamos la librer\'ia de \texttt{numpy}. Corremos el programa (antes activando el virtual environment, e instalando numpy mediante \texttt{pip install numpy} si es necesario) mediante
		      \begin{center}
			      \texttt{python3 p1\textunderscore verify.py t5a/Eigen\textunderscore125x125.txt 7 0}
		      \end{center}
		      Este nos genera los valores
		      \begin{table}[H]
			      \begin{center}
				      \begin{tabular}{|l|l|}
					      \hline
					      n & Eigenvalue \\
					      \hline
					      1 & 0.000002\\
					      \hline
					      2 & 0.000008\\
					      \hline
					      3 &0.000017\\
					      \hline
					      4 & 0.000018\\
					      \hline
					      5 & 0.000026\\
					      \hline
					      6 & 0.000036\\
					      \hline
					      7 & 0.000038\\
					      \hline
				      \end{tabular}
			      \end{center}
			      \caption{Eigenvalores de Eigen\_125x125.txt verificados con numpy.}\label{tab:invpy125x125}
		      \end{table}
          Para esta matriz, usamos una tolerancia de menor magnitud para aproximar los eigenvalores de manera m\'as cercana a los reales. Con esto, verificamos que los calculados son correctos.
	\end{enumerate}

\end{solution}
%end solution 1

\bibliographystyle{plain}
\bibliography{main}

\end{document}
