\documentclass{article}
\usepackage[yyyymmdd]{datetime}
\usepackage[nottoc]{tocbibind}
\usepackage{pgfplots}
\pgfplotsset{compat=1.18}
\usepackage{url}
\usepackage{array}
\usepackage{float}
\usepackage[inline,shortlabels]{enumitem}
\usepackage{hhline}
\usepackage{multirow}
\usepackage{geometry}
\usepackage[dvipsnames]{xcolor}
\usepackage{colortbl}
\usepackage[framemethod=TikZ]{mdframed}
\usepackage{amsfonts}
\usepackage{amsmath}
\usepackage{tikz}
\usepackage{amsthm}
\newtheoremstyle{problemstyle}{3pt}{3pt}{\normalfont}{}{\bfseries}{\normalfont\bfseries:}{.5em}{}
\theoremstyle{problemstyle}
\newmdtheoremenv[
  linewidth=1pt,
  linecolor=RoyalBlue,
  backgroundcolor=RoyalBlue!10,
  roundcorner=5pt,
  innertopmargin=6pt,
  innerbottommargin=6pt,
  innerleftmargin=6pt,
  innerrightmargin=6pt,
  nobreak=true
]{problem}{Problem}

% Example
\newmdtheoremenv[
  linewidth=1pt,
  linecolor=ForestGreen,
  backgroundcolor=ForestGreen!10,
  roundcorner=5pt,
  nobreak=true
]{example}{Example}

% Theorem
\newmdtheoremenv[
  linewidth=1pt,
  linecolor=BrickRed,
  backgroundcolor=BrickRed!10,
  roundcorner=5pt,
  nobreak=true
]{theorem}{Theorem}

% Remark
\newmdtheoremenv[
  linewidth=1pt,
  linecolor=Goldenrod,
  backgroundcolor=Goldenrod!10,
  roundcorner=5pt,
  nobreak=true
]{remark}{Remark}

% Solution
\newenvironment{solution}{%
  \begin{mdframed}[linewidth=0.8pt,linecolor=Gray,backgroundcolor=Gray!5,roundcorner=5pt]%
  \noindent\textbf{Solution.}%
}{%
\hfill $ \diamond $ 
  \end{mdframed}%
}

\usepackage{listings}

\lstset{breaklines=true}
\definecolor{vscode-blue}{RGB}{64, 116, 160}   % Keywords
\definecolor{vscode-green}{RGB}{95, 203, 114}   % Strings
\definecolor{vscode-gray}{RGB}{128, 128, 128}   % Comments
\definecolor{vscode-orange}{RGB}{206, 145, 120} % Numbers/Constants
\definecolor{vscode-purple}{RGB}{204, 124, 255} % Functions/Types
\definecolor{vscode-cyan}{RGB}{155, 199, 217}   % Preprocessor

\lstdefinestyle{visual-studio}{
    language=C,
    backgroundcolor=\color{Gray!10},
    commentstyle=\color{vscode-gray}\ttfamily,
    keywordstyle=\color{vscode-blue}\ttfamily,
    stringstyle=\color{vscode-green}\ttfamily,
    numberstyle=\color{vscode-orange}\ttfamily,
    identifierstyle=\color{black}\ttfamily,
    breaklines=true,
    showstringspaces=false,
    tabsize=4,
    % Additional settings for more Visual Studio feel
    emph=[1]{void, int, double, float, char, bool, short, long, signed, unsigned, class, struct, enum, union, typedef, template}, % Common types
    emphstyle=[1]\color{vscode-purple},
    emph=[2]{printf, scanf, cin, cout, return, new, delete, if, else, while, do, for, switch, case, default, break, continue, goto, throw, try, catch, const, static, extern, volatile, register, restrict, inline, explicit, virtual, friend, namespace, using, private, protected, public, operator}, % Standard keywords
    emphstyle=[2]\color{vscode-blue},
            keywordstyle=[2]\color{vscode-cyan}\ttfamily\bfseries,
    morekeywords={\#include, \#define, \#ifdef, \#ifndef, \#endif, \#pragma},
    basicstyle=\small
}

% Make header with name and date etc.
\usepackage{fancyhdr}
\lhead{Pedro D. Llerenas\\M\'etodos Num\'ericos I}
\rhead{\today\\Tarea III}
\thispagestyle{fancy}

\usepackage[utf8]{inputenc}
\setlength{\parindent}{0pt} % Don't indent new paragraphs
\setlength{\headheight}{24pt} 

\newcommand{\Z}{\mathbb Z}
\newcommand{\Q}{\mathbb Q}
\newcommand{\R}{\mathbb R}
\newcommand{\C}{\mathbb C}
\newcommand{\N}{\mathbb N}

\begin{document}
\section{Sistemas de ecuaciones lineales}\label{sec:sistemas_de_ecuaciones_lineales} % (fold)


\begin{problem}
Desarrollar un programa que resuelva un sistema de ecuaciones de la forma $\mathbf{Dx} = \mathbf{b}$, donde $ \mathbf{D} $ es una matriz diagonal. (\textbf{1.5 puntos})
\end{problem}
\begin{solution}
	Sea $ \mathbf{D} $ la matriz diagonal con valores diagonales
	\begin{align*}
		4.48\quad 6.48\quad 9.65\quad 2.40\quad 8.62\quad 5.58\quad 9.01\quad 1.46\quad 5.96\quad 2.50\quad 8.63\quad 5.46\quad 4.78\quad 9.13\quad 5.90
	\end{align*}
	y sea
	\begin{align*}
		\mathbf{b} = \begin{bmatrix}
			             29.68 \\ 31.59 \\ 47.36 \\ 13.02 \\ 31.68 \\ 32.97 \\ 78.34 \\ 12.80 \\ 14.56 \\ 2.88 \\ 62.87 \\ 50.29 \\ 6.40 \\ 32.57 \\ 45.27
		             \end{bmatrix}
	\end{align*}

	Queremos solucionar el sistema
	$$
		\left[
			\begin{array}{cccc}
				4.48            &      &        & \mbox{\Large 0} \\
				                & 6.48 &        &                 \\
				                &      & \ddots &                 \\
				\mbox{\Large 0} &      &        & 5.90            \\
			\end{array}
			\right]
		\left[
			\begin{array}{c}
				x_1    \\
				x_2    \\
				\vdots \\
				x_{15} \\
			\end{array}
			\right]
		=
		\left[
			\begin{array}{c}
				29.68  \\
				31.59  \\
				\vdots \\
				45.27  \\
			\end{array}
			\right]
	$$

	Usando \texttt{make run ARGS="eq/D.txt eq/b\_d.txt D"} obtenemos el vector soluci\'on $ \mathbf{x} $:
	\begin{align*}
		\mathbf{x} = \begin{bmatrix}
			             6.63 \\ 4.88 \\ 4.91 \\ 5.42 \\ 3.68 \\ 5.91 \\ 8.69 \\ 8.79 \\ 2.45 \\ 1.15 \\ 7.29 \\ 9.21 \\ 1.34 \\ 3.57 \\ 7.68
		             \end{bmatrix}
	\end{align*}

\end{solution}

\begin{problem}
Desarrollar un programa que resuelva un sistema de ecuaciones de la forma $\mathbf{Ux} = \mathbf{b}$, donde $ \mathbf{U} $ es una matriz triangular superior. (\textbf{2.5 puntos})
\end{problem}
\begin{solution}
	Con la matriz $ \mathbf{U} $ dada por el archivo \texttt{U.txt} (demsiada grande para mostrar aqu\'i) y el vector \texttt{b\_u.txt}, obtenemos que el $ \mathbf{x} $ tal que $ \mathbf{Ux} = \mathbf{b} $ es
	\begin{align*}
		\mathbf{x} =
		\begin{bmatrix}
			6.63 \\ 2.03 \\ 6.60 \\ 7.52 \\ 4.58 \\ 1.32 \\ 5.46 \\ 2.44 \\ 7.27 \\ 7.25 \\ 5.75 \\ 1.57 \\ 9.63 \\ 6.62 \\ 2.34
		\end{bmatrix}.
	\end{align*}
	Para ejecutar el programa que lo soluciona, usar \texttt{make run ARGS="eq/U.txt eq/b\_u.txt U"}.

\end{solution}

\begin{problem}
Desarrollar un programa que resuelva un sistema de ecuaciones de la forma $\mathbf{Lx} = \mathbf{b}$, donde $ \mathbf{L} $ es una matriz triangular inferior. (\textbf{2.5 puntos})
\end{problem}

\begin{solution}
	Con la matriz $ \mathbf{L} $ dada por el archivo \texttt{L.txt} (demsiada grande para mostrar aqu\'i) y el vector \texttt{b\_l.txt}, obtenemos que el $ \mathbf{x} $ tal que $ \mathbf{Ux} = \mathbf{b} $ es
	\begin{align*}
		\mathbf{x} =
		\begin{bmatrix}
			6.63 \\ 4.92 \\ 7.00 \\ 9.72 \\ 7.10 \\ 3.45 \\ 7.80 \\ 3.70 \\ 2.82 \\ 8.90 \\ 3.07 \\ 1.95 \\ 2.14 \\ 2.83 \\ 2.20
		\end{bmatrix}.
	\end{align*}
	Para ejecutar el programa que lo soluciona, usar \texttt{make run ARGS="eq/L.txt eq/b\_l.txt L"}.
\end{solution}


\begin{problem}
Describe el algoritmo de eliminaci\'on Gaussiana. (\textbf{0.5 puntos})
\end{problem}
\begin{solution}
	El algoritmo consiste en realizar operacions b\'asicas de matrices para convertir una matriz de la forma
	\begin{align*}
		\begin{bmatrix}
			a_{11} & \cdots & a_{1n} \\
			\vdots & \ddots & \vdots \\
			a_{n1} & \cdots & a_{nn}
		\end{bmatrix}
	\end{align*}
	en una matriz triangular superior
	\begin{align*}
		\begin{bmatrix}
			u_{11} & \cdots & u_{1n} \\
			\vdots & \ddots & \vdots \\
			0      & \cdots & u_{nn}
		\end{bmatrix}.
	\end{align*}
	Con operaciones b\'asicas, nos referimos a:
	\begin{itemize}
		\item Cambio de posici\'on de renglones,
		\item Producto escalar no cero de un rengl\'on,
		\item Sumarle un m\'ultiplo escalar de un rengl\'on a otro.
	\end{itemize}
	Espec\'ificamente, si tenemos un sistema de ecuaciones lineales
	\begin{align*}
		\arraycolsep=1pt
		\begin{array}{cccc}
			E_{1} : & a_{11}x_{1} + a_{12}x_{2} + \cdots + a_{1n}x_{n} & =      & b_{1}, \\
			E_{2} : & a_{21}x_{1} + a_{22}x_{2} + \cdots + a_{2n}x_{n} & =      & b_{2}, \\
			        & \vdots                                           & \vdots &        \\
			E_{n} : & a_{n1}x_{1} + a_{n2}x_{2} + \cdots + a_{nn}x_{n} & =      & b_{n},
		\end{array}
	\end{align*}
	consideramos la matriz aumentada
	\begin{align*}
		A_1 = [A,\mathbf{b}] =
		\begin{bmatrix}
			\begin{array}{cccc|c}
				a_{11} & a_{12} & \cdots & a_{1n} & a_{1,n+1} \\
				a_{21} & a_{22} & \cdots & a_{2n} & a_{2,n+1} \\
				\vdots & \vdots & \ddots & \vdots & \vdots    \\
				a_{n1} & a_{n2} & \cdots & a_{nn} & a_{n,n+1}
			\end{array}
		\end{bmatrix}
	\end{align*}
	donde $ a_{i, n+1} = b_{i} $.

  Si $ a_{11} \neq 0 $, realizamos \cite{Burden2017}
	\begin{align*}
		(E_j - (a_{j1}/a_{11})E_1) \to (E_j)\quad \text{para cada } j = 2,3,\dots,n.
	\end{align*}
	Es decir, la ecuaci\'on $ j $ es reemplazada por la combinaci\'on lineal de ecuaciones del lado izquierdo.
	Estas operaciones tiene el fin de eliminar el t\'ermino $ x_1 $ de todas las filas excepto la primera. Es decir, obtenemos la matriz
	\begin{align*} A_2 =
		\begin{bmatrix}
			\begin{array}{cccc|c}
				a_{11} & a_{12} & \cdots & a_{1n} & a_{1,n+1} \\
				0      & a_{22} & \cdots & a_{2n} & a_{2,n+1} \\
				\vdots & \vdots & \ddots & \vdots & \vdots    \\
				0      & a_{n2} & \cdots & a_{nn} & a_{n,n+1}
			\end{array}
		\end{bmatrix},
	\end{align*}
	donde los $ a_{ij} $ son posiblemente diferentes a los escritos en la primer matriz. Secuencialmente para $ i = 2,3,\dots, n-1 $, realizamos las operaciones
	\begin{align*}
		(E_j - (a_{ji}/a_{ii})E_j) \to (E_j)\quad \text{para cada } j = i+1, i+2,\dots, n.
	\end{align*}
	mientras $ a_{ii}\neq 0 $. Si en alg\'un momento esta entrada es cero, buscamos una fila que pueda reemplazarla. Si no existe, el algoritmo se detiene y determinamos que al menos una columna (o fila) es linealmente dependiente de otra, por lo que el sistema no tiene soluci\'on \'unica. Si logramos continuar para todos los \'indices, obtenemos la matriz
	\begin{align*} A_m =
		\begin{bmatrix}
			\begin{array}{cccc|c}
				a_{11} & a_{12} & \cdots & a_{1n} & a_{1,n+1} \\
				0      & a_{22} & \cdots & a_{2n} & a_{2,n+1} \\
				\vdots & \vdots & \ddots & \vdots & \vdots    \\
				0      & 0      & \cdots & a_{nn} & a_{n,n+1}
			\end{array}
		\end{bmatrix},
	\end{align*}
	donde nuevamente, los $ a_{ij} $ son posiblemente distintos a los originalmente dados. Entonces, tenemos un sistema lineal triangular
	\begin{align*}
		\arraycolsep=1pt
		\begin{array}{rcr}
			a_{11}x_{1} + a_{12}x_{2} + \cdots + a_{1n}x_{n} & =      & b_{1}, \\
			a_{22}x_{2} + \cdots + a_{2n}x_{n}               & =      & b_{2}, \\
			                                                 & \vdots &        \\
			a_{nn}x_{n}                                      & =      & b_{n},
		\end{array}.
	\end{align*}
\end{solution}
\begin{problem}
Desarrollar un programa que resuelva un sistema de ecuaciones de la forma $ \mathbf{Ax} = \mathbf{b} $ mediante el m\'etodo de eliminaci\'on Gaussiana. Una vez que realices la eliminaci\'on Gaussiana puedes utilizar el m\'etodo anteriormente programado para resolver el sistema equivalente $ \mathbf{Ux} = \mathbf{b} $. (\textbf{3.0 puntos})
\end{problem}
\begin{solution}
  Utilizando $ \mathbf{A} $ como la matriz del archivo \texttt{A.txt} y $ \mathbf{b} $ como el vector del archivo \texttt{b.txt}, obtenemos que la soluci\'on a $ \mathbf{Ax} = \mathbf{b} $ corresponde a
    \begin{align*}
      \mathbf{x} =\begin{bmatrix}
    423.72 \\
    95.59 \\
    154.56 \\
    19.78 \\
    190.43 \\
    -243.85 \\
    51.78 \\
    60.59 \\
    -77.99 \\
    -157.03 \\
    -157.78 \\
    -51.55 \\
    -48.22 \\
    -29.28 \\
    -161.45
\end{bmatrix} 
    \end{align*}
  Para ejecutar el codigo, usar \texttt{make run\_gaussian ARGS="eq/A.txt eq/b.txt"}.

\end{solution}
\bibliographystyle{plain}
\bibliography{main}

% section Sistemas de ecuaciones lineales (end)
\end{document}

