\documentclass{article}
\usepackage[yyyymmdd]{datetime}
\usepackage[nottoc]{tocbibind}
\usepackage{pgfplots}
\pgfplotsset{compat=1.18}
\usepackage{url}
\usepackage{array}
\usepackage{float}
\usepackage[inline,shortlabels]{enumitem}
\usepackage{hhline}
\usepackage{multirow}
\usepackage{geometry}
\usepackage[dvipsnames]{xcolor}
\usepackage{colortbl}
\usepackage[framemethod=TikZ]{mdframed}
\usepackage{amsfonts}
\usepackage{amsmath}
\usepackage{tikz}
\usepackage{amsthm}
\newtheoremstyle{problemstyle}{3pt}{3pt}{\normalfont}{}{\bfseries}{\normalfont\bfseries:}{.5em}{}
\theoremstyle{problemstyle}
\newmdtheoremenv[
  linewidth=1pt,
  linecolor=RoyalBlue,
  backgroundcolor=RoyalBlue!10,
  roundcorner=5pt,
  innertopmargin=6pt,
  innerbottommargin=6pt,
  innerleftmargin=6pt,
  innerrightmargin=6pt,
  nobreak=true
]{problem}{Problem}

% Example
\newmdtheoremenv[
  linewidth=1pt,
  linecolor=ForestGreen,
  backgroundcolor=ForestGreen!10,
  roundcorner=5pt,
  nobreak=true
]{example}{Example}

% Theorem
\newmdtheoremenv[
  linewidth=1pt,
  linecolor=BrickRed,
  backgroundcolor=BrickRed!10,
  roundcorner=5pt,
  nobreak=true
]{theorem}{Theorem}

% Remark
\newmdtheoremenv[
  linewidth=1pt,
  linecolor=Goldenrod,
  backgroundcolor=Goldenrod!10,
  roundcorner=5pt,
  nobreak=true
]{remark}{Remark}

% Solution
\newenvironment{solution}{%
  \begin{mdframed}[linewidth=0.8pt,linecolor=Gray,backgroundcolor=Gray!5,roundcorner=5pt]%
  \noindent\textbf{Solution.}%
}{%
\hfill $ \diamond $ 
  \end{mdframed}%
}

\usepackage{listings}

\lstset{breaklines=true}
\definecolor{vscode-blue}{RGB}{64, 116, 160}   % Keywords
\definecolor{vscode-green}{RGB}{95, 203, 114}   % Strings
\definecolor{vscode-gray}{RGB}{128, 128, 128}   % Comments
\definecolor{vscode-orange}{RGB}{206, 145, 120} % Numbers/Constants
\definecolor{vscode-purple}{RGB}{204, 124, 255} % Functions/Types
\definecolor{vscode-cyan}{RGB}{155, 199, 217}   % Preprocessor

\lstdefinestyle{visual-studio}{
    language=C,
    backgroundcolor=\color{Gray!10},
    commentstyle=\color{vscode-gray}\ttfamily,
    keywordstyle=\color{vscode-blue}\ttfamily,
    stringstyle=\color{vscode-green}\ttfamily,
    numberstyle=\color{vscode-orange}\ttfamily,
    identifierstyle=\color{black}\ttfamily,
    breaklines=true,
    showstringspaces=false,
    tabsize=4,
    % Additional settings for more Visual Studio feel
    emph=[1]{void, int, double, float, char, bool, short, long, signed, unsigned, class, struct, enum, union, typedef, template}, % Common types
    emphstyle=[1]\color{vscode-purple},
    emph=[2]{printf, scanf, cin, cout, return, new, delete, if, else, while, do, for, switch, case, default, break, continue, goto, throw, try, catch, const, static, extern, volatile, register, restrict, inline, explicit, virtual, friend, namespace, using, private, protected, public, operator}, % Standard keywords
    emphstyle=[2]\color{vscode-blue},
            keywordstyle=[2]\color{vscode-cyan}\ttfamily\bfseries,
    morekeywords={\#include, \#define, \#ifdef, \#ifndef, \#endif, \#pragma},
    basicstyle=\small
}

% Make header with name and date etc.
\usepackage{fancyhdr}
\lhead{Pedro D. Llerenas\\M\'etodos Num\'ericos I}
\rhead{\today\\Tarea IV}
\thispagestyle{fancy}

\usepackage[utf8]{inputenc}
\setlength{\parindent}{0pt} % Don't indent new paragraphs
\setlength{\headheight}{24pt} 

\newcommand{\Z}{\mathbb Z}
\newcommand{\Q}{\mathbb Q}
\newcommand{\R}{\mathbb R}
\newcommand{\C}{\mathbb C}
\newcommand{\N}{\mathbb N}

\begin{document}
\section*{Factoriazaci\'on de matrices y m\'etodos iterativos}\label{sec:factoriazaci_on_de_matrices_y_m_etodos_iterativos} % (fold)


%begin problem 1
\begin{problem}
Desarrollar un programa para la soluci\'on a un sistema de ecuaciones utilizando factorizaci\'on $ \mathbf{A} = \mathbf{LU} $ usando el método de \textit{Crout} con $diag(\mathbf{U}) = \{1, 1, . . . , 1\}$. \textbf{(1.5 puntos)}
\end{problem}
%end problem 1

%begin solution 1
\begin{solution}
\end{solution}
%end solution 1

%begin problem 2
\begin{problem}
  Desarrollar de forma escrita las expresiones para resolver el sistema de ecuaciones utilizando la factorizaci\'on $ \mathbf{A} = \mathbf{LU} $ usando el m\'etodo \textit{Doolittle} de la forma $ \mathbf{A} = \mathbf{LU} $, con $diag(\mathbf{L}) = \{1, 1, . . . , 1\}$. (No es necesario desarrollar el programa).\textbf{(0.5 puntos)}
\end{problem}
%end problem 2

%begin solution 2
\begin{solution}
\end{solution}
%end solution 2

%begin problem 3
\begin{problem}
  Desarrollar un programa para la soluci\'on a un sistema de ecuaciones utilizando factorizaci\'on $ \mathbf{A} = \mathbf{LL^T} $ (M\'etodo de Cholesky). \textbf{(1.5 puntos)}
\end{problem}
%end problem 3

%begin solution 3
\begin{solution}
\end{solution}
%end solution 3

%begin problem 4
\begin{problem}
  Resolver la ecuaci\'on de calor en 1D de la forma:
  \begin{align*}
    K \frac{\partial^2\phi}{\partial x^2} + Q = 0
  \end{align*}
  con las siguientes condiciones de contorno, en una l\'inea de longitud $ L$, $ K = 1 $, $ Q = 5 $
  \begin{align*}
    \phi_0 &= 0\\
    \phi_n &= 100
  \end{align*}
  Para resolver el sistema de ecuaciones, deber\'a de usar el solver de una matriz tridiagonal sim\'etrica usando el m\'etodo de Cholesky. La matriz se guarda en 2 vectores. \textbf{(3.5 puntos)}

\end{problem}
%end problem 4

%begin solution 4
\begin{solution}
\end{solution}
%end solution 4

%begin problem 5
\begin{problem}
  Desarrollar un programa que resuelva un sistema de ecuaciones de la forma $ \mathbf{Ax} = \mathbf{b} $ utilizando el método iterativo de \textit{Jacobi}. \textbf{(1.5 puntos)}
\end{problem}
%end problem 5

%begin solution 5
\begin{solution}
\end{solution}
%end solution 5

%begin problem 6
\begin{problem}
  Desarrollar un programa que resuelva un sistema de ecuaciones de la forma $ \mathbf{Ax} = \mathbf{b} $ utilizando el método iterativo de \textit{Gauss-Seidel}. \textbf{(1.5 puntos)}
\end{problem}
%end problem 6

%begin solution 6
\begin{solution}
\end{solution}
%end solution 6
% section Factoriazaci\'on de matrices y m\'etodos iterativos (end)


\bibliographystyle{plain}
\bibliography{main}

\end{document}

