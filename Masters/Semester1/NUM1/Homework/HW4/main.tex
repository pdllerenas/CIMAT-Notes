\documentclass{article}
\usepackage[yyyymmdd]{datetime}
\usepackage[nottoc]{tocbibind}
\usepackage{pgfplots}
\pgfplotsset{compat=1.18}
\usepackage{url}
\usepackage{array}
\usepackage{bm}
\usepackage{float}
\usepackage[inline,shortlabels]{enumitem}
\usepackage{hhline}
\usepackage{multirow}
\usepackage{geometry}
\usepackage[dvipsnames]{xcolor}
\usepackage{colortbl}
\usepackage[framemethod=TikZ]{mdframed}
\usepackage{amsfonts}
\usepackage{amsmath}
\usepackage{tikz}
\usepackage{amsthm}
\newtheoremstyle{problemstyle}{3pt}{3pt}{\normalfont}{}{\bfseries}{\normalfont\bfseries:}{.5em}{}
\theoremstyle{problemstyle}
\newmdtheoremenv[
  linewidth=1pt,
  linecolor=RoyalBlue,
  backgroundcolor=RoyalBlue!10,
  roundcorner=5pt,
  innertopmargin=6pt,
  innerbottommargin=6pt,
  innerleftmargin=6pt,
  innerrightmargin=6pt,
  nobreak=true
]{problem}{Problem}

% Example
\newmdtheoremenv[
  linewidth=1pt,
  linecolor=ForestGreen,
  backgroundcolor=ForestGreen!10,
  roundcorner=5pt,
  nobreak=true
]{example}{Example}

% Theorem
\newmdtheoremenv[
  linewidth=1pt,
  linecolor=BrickRed,
  backgroundcolor=BrickRed!10,
  roundcorner=5pt,
  nobreak=true
]{theorem}{Theorem}

% Remark
\newmdtheoremenv[
  linewidth=1pt,
  linecolor=Goldenrod,
  backgroundcolor=Goldenrod!10,
  roundcorner=5pt,
  nobreak=true
]{remark}{Remark}

% Solution
\newenvironment{solution}{%
  \begin{mdframed}[linewidth=0.8pt,linecolor=Gray,backgroundcolor=Gray!5,roundcorner=5pt]%
  \noindent\textbf{Solution.}%
}{%
\hfill $ \diamond $ 
  \end{mdframed}%
}

\usepackage{listings}

\lstset{breaklines=true}
\definecolor{vscode-blue}{RGB}{64, 116, 160}   % Keywords
\definecolor{vscode-green}{RGB}{95, 203, 114}   % Strings
\definecolor{vscode-gray}{RGB}{128, 128, 128}   % Comments
\definecolor{vscode-orange}{RGB}{206, 145, 120} % Numbers/Constants
\definecolor{vscode-purple}{RGB}{204, 124, 255} % Functions/Types
\definecolor{vscode-cyan}{RGB}{155, 199, 217}   % Preprocessor

\lstdefinestyle{visual-studio}{
    language=C,
    backgroundcolor=\color{Gray!10},
    commentstyle=\color{vscode-gray}\ttfamily,
    keywordstyle=\color{vscode-blue}\ttfamily,
    stringstyle=\color{vscode-green}\ttfamily,
    numberstyle=\color{vscode-orange}\ttfamily,
    identifierstyle=\color{black}\ttfamily,
    breaklines=true,
    showstringspaces=false,
    tabsize=4,
    % Additional settings for more Visual Studio feel
    emph=[1]{void, int, double, float, char, bool, short, long, signed,
    unsigned, class, struct, enum, union, typedef, template}, % Common types
    emphstyle=[1]\color{vscode-purple},
    emph=[2]{printf, scanf, cin, cout, return, new, delete, if, else, while,
    do, for, switch, case, default, break, continue, goto, throw, try, catch,
  const, static, extern, volatile, register, restrict, inline, explicit,
virtual, friend, namespace, using, private, protected, public, operator}, %
Standard keywords
    emphstyle=[2]\color{vscode-blue},
            keywordstyle=[2]\color{vscode-cyan}\ttfamily\bfseries,
    morekeywords={\#include, \#define, \#ifdef, \#ifndef, \#endif, \#pragma},
    basicstyle=\small
}

% Make header with name and date etc.
\usepackage{fancyhdr}
\lhead{Pedro D. Llerenas\\M\'etodos Num\'ericos I}
\rhead{\today\\Tarea IV}
\thispagestyle{fancy}

\usepackage[utf8]{inputenc}
\setlength{\parindent}{0pt} % Don't indent new paragraphs
\setlength{\headheight}{24pt} 

\newcommand{\Z}{\mathbb Z}
\newcommand{\Q}{\mathbb Q}
\newcommand{\R}{\mathbb R}
\newcommand{\C}{\mathbb C}
\newcommand{\N}{\mathbb N}

\begin{document}
\section*{Factoriazaci\'on de matrices y m\'etodos iterativos}\label{sec:factoriazaci_on_de_matrices_y_m_etodos_iterativos} % (fold)


%begin problem 1
\begin{problem}
Desarrollar un programa para la soluci\'on a un sistema de ecuaciones
utilizando factorizaci\'on $ \mathbf{A} = \mathbf{LU} $ usando el método de
\textit{Crout} con $diag(\mathbf{U}) = \{1, 1, . . . , 1\}$. \textbf{(1.5
	puntos)}
\end{problem}
%end problem 1

%begin solution 1
\begin{solution}
	En este caso, utilizamos $ \mathbf{A} $ como \texttt{A.txt} y $ \mathbf{b} $ como \texttt{b.txt}. Para ejecutar el programa que resuelve este sistema usando el m\'etodo de Crout, usar

	\begin{center}
		\texttt{make run-p1 ARGS="lin\_sys/A.txt lin\_sys/b.txt"}
	\end{center}
	Esto nos genera 3 archivos en el directorio \texttt{lin\_sys}: \texttt{crout\_L.txt}, \texttt{crout\_U.txt} y \texttt{crout\_x.txt}, ese decir, la descomposici\'on de Crout y la soluci\'on del sistema.
\end{solution}
%end solution 1

%begin problem 2
\begin{problem}
Desarrollar de forma escrita las expresiones para resolver el sistema de
ecuaciones utilizando la factorizaci\'on $ \mathbf{A} = \mathbf{LU} $ usando el
m\'etodo \textit{Doolittle} de la forma $ \mathbf{A} = \mathbf{LU} $, con
$diag(\mathbf{L}) = \{1, 1, . . . , 1\}$. (No es necesario desarrollar el
programa).\textbf{(0.5 puntos)}
\end{problem}
%end problem 2

%begin solution 2
\begin{solution}
	Para el m\'etodo de Doolittle comenzamos con
	\begin{align*}
		l_{ii} = 1.
	\end{align*}
	Entonces, los t\'erminos de $ \mathbf{U} $ est\'an dados por
	\begin{align*}
		u_{ij} =
		\begin{cases}
			a_{ij} - \sum_{k=0}^{i-1}l_{ik}u_{kj} & \text{ si } i>0,   \\
			a_{ij}                                & \text{ si } i = 0.
		\end{cases}
	\end{align*}
	Mientras que los de $ \mathbf{L} $ son dados por
	\begin{align*}
		l_{ij} =
		\begin{cases}
			\dfrac{a_{ij} - \sum_{k=0}^{i-1}l_{ik}u_{kj}}{u_{jj}} & \text{ si } i>0,   \\
			\dfrac{a_{ij}}{u_{jj}}                                & \text{ si } i = 0.
		\end{cases}
	\end{align*}
\end{solution}
%end solution 2
\pagebreak
%begin problem 3
\begin{problem}
Desarrollar un programa para la soluci\'on a un sistema de ecuaciones
utilizando factorizaci\'on $ \mathbf{A} = \mathbf{LL^T} $ (M\'etodo de
Cholesky). \textbf{(1.5 puntos)}
\end{problem}
%end problem 3

%begin solution 3
\begin{solution}
	Consideremos la matriz sim\'etrica definida positiva $ \mathbf{A} $ dada por \texttt{SPD.txt}, y el vector $ \mathbf{b} $ dado por \texttt{b\_spd.txt}. Para ejecutar el programa que resuelve esto mediante la factorizaci\'on de Cholesky, usamos

	\begin{center}
		\texttt{make run-p3 ARGS="lin\_sys/SPD.txt lin\_sys/b\_spd.txt"}
	\end{center}
	Esto genera 3 archivos en \texttt{lin\_sys/}: \texttt{cholesky\_L.txt}, \texttt{cholesky\_LT.txt} y \texttt{cholesky\_x.txt}.
\end{solution}
%end solution 3

%begin problem 4
\begin{problem}
Resolver la ecuaci\'on de calor en 1D de la forma:
\begin{align*}
	K \frac{\partial^2\phi}{\partial x^2} + Q = 0
\end{align*}
con las siguientes condiciones de contorno, en una l\'inea de longitud $ L$,
$ K = 1 $, $ Q = 5 $
\begin{align*}
	\phi_0 & = 0   \\
	\phi_n & = 100
\end{align*}
Para resolver el sistema de ecuaciones, deber\'a de usar el solver de una
matriz tridiagonal sim\'etrica usando el m\'etodo de Cholesky. La matriz se
guarda en 2 vectores. \textbf{(3.5 puntos)}

\end{problem}
%end problem 4

%begin solution 4
\begin{solution}
	Para ejecutar el programa que resuelve la ecuaci\'on de calor, usamos

	\begin{center}
		\texttt{make run-p4 ARGS="1.0 1.0 5.0 0.0 100.0 100"}
	\end{center}

	donde los argumentos (en ese \'orden) son los siguientes:
	\begin{enumerate}
		\item $ L = 1.0 $,
		\item $ K = 1.0 $,
		\item $ Q = 1.0 $,
		\item $ \phi_0 = 0.0 $,
		\item $ \phi_{n} = 100.0 $,
		\item $ n = 100 $.
	\end{enumerate}
	Esto nos genera un archivo \texttt{phi.txt}, conteniendo las $ n = 100 $ inc\'ognitas de la barra.
\end{solution}
%end solution 4

%begin problem 5
\begin{problem}
Desarrollar un programa que resuelva un sistema de ecuaciones de la forma $
	\mathbf{Ax} = \mathbf{b} $ utilizando el método iterativo de \textit{Jacobi}.
\textbf{(1.5 puntos)}
\end{problem}
%end problem 5

%begin solution 5
\begin{solution}
	Primero, probamos $ \mathbf{A} $ con valores \texttt{A\_small.txt} y $ \mathbf{b} $ con valores \texttt{b\_small.txt} y corremos el programa usando
	\begin{center}
    \texttt{make run-p5 ARGS="lin\_sys/A\_small.txt lin\_sys/b\_small.txt
    lin\_sys/x0\_small.txt 0.00001 100"}
	\end{center}
	Esto nos genera el archivo del vector soluci\'on \texttt{jacobi\_x.txt}, que tiene valores
	\begin{align*}
		\mathbf{x} =
		\begin{bmatrix}
			3.0  \\
			-2.5 \\
			7.0
		\end{bmatrix}
	\end{align*}
	La convergencia es la siguiente:
	\begin{table}[H]
		\begin{center}
			\begin{tabular}[H]{c|c}
				n & $ \lVert x_i - x_{i-1}\rVert $ \\
				\hline
				1 & 54.426089342403642             \\
				2 & 0.153501234126984              \\
				3 & 0.000104009711195              \\
				4 & 0.000000375866745
			\end{tabular}
		\end{center}
	\end{table}

	Ahora, si consideramos \texttt{A\_big.txt} y \texttt{b\_big.txt} y lo ejecuramos usando
	\begin{center}
		\texttt{make run-p5 ARGS="lin\_sys/A\_big.txt lin\_sys/b\_big.txt lin\_sys/x0\_big.txt 0.00001 1000"}
	\end{center}
	obtenemos el mismo archivo, pero ahora con los valores soluci\'on al sistema
	de $ 125\times 125 $, y la convergencia se ve de la siguiente manera:
	\begin{table}[H]
		\begin{center}
			\begin{tabular}[H]{c|c}
				n        & $ \lVert x_i - x_{i-1}\rVert $ \\
				\hline
				1        & 13.139842374895840             \\
				2        & 0.275904211875408              \\
				3        & 0.137594762031032              \\
				$\vdots$ & $ \vdots $                     \\
				373      & 0.000010143756020              \\
				374      & 0.000010009285854              \\
				375      & 0.000009876598130              \\
			\end{tabular}
		\end{center}
	\end{table}

\end{solution}
%end solution 5

%begin problem 6
\begin{problem}
Desarrollar un programa que resuelva un sistema de ecuaciones de la forma $
	\mathbf{Ax} = \mathbf{b} $ utilizando el método iterativo de
\textit{Gauss-Seidel}. \textbf{(1.5 puntos)}
\end{problem}
%end problem 6

%begin solution 6
\begin{solution}
	Primero, probamos $ \mathbf{A} $ con valores \texttt{A\_small.txt} y $ \mathbf{b} $ con valores \texttt{b\_small.txt} y corremos el programa usando
	\begin{center}
		\texttt{make run-p6 ARGS="lin\_sys/A\_small.txt lin\_sys/b\_small.txt lin\_sys/x0\_small.txt 0.00001 100"}
	\end{center}
	Esto nos genera el archivo del vector soluci\'on \texttt{gauss\_seidel\_x.txt}, que tiene valores
	\begin{align*}
		\mathbf{x} =
		\begin{bmatrix}
			3.0  \\
			-2.5 \\
			7.0
		\end{bmatrix}
	\end{align*}
	Con la siguiente convergencia:
	\begin{table}[H]
		\begin{center}
			\begin{tabular}[H]{c|c}
				n & $ \lVert x_i - x_{i-1}\rVert $ \\
				\hline
				1 & 53.073937274013616             \\
				2 & 0.139895692236055              \\
				3 & 0.000067908604963              \\
				4 & 0.000000000748281
			\end{tabular}
		\end{center}
	\end{table}

	Ahora, si consideramos \texttt{A\_big.txt} y \texttt{b\_big.txt} y lo ejecuramos usando
	\begin{center}
		\texttt{make run-p6 ARGS="lin\_sys/A\_big.txt lin\_sys/b\_big.txt lin\_sys/x0\_big.txt 0.00001 1000"}
	\end{center}
	obtenemos el mismo archivo, pero ahora con los valores soluci\'on al sistema de $ 125\times 125 $.
	\begin{table}[H]
		\begin{center}
			\begin{tabular}[H]{c|c}
				n        & $ \lVert x_i - x_{i-1}\rVert $ \\
				\hline
        1& 13.713414633215809\\
        2& 0.492334502610899\\
        3& 0.223358458755092\\
				$\vdots$ & $ \vdots $                     \\
        238& 0.000010444020240\\
        239& 0.000010169122768\\
        240& 0.000009901460824
			\end{tabular}
		\end{center}
	\end{table}
\end{solution}
%end solution 6
% section Factoriazaci\'on de matrices y m\'etodos iterativos (end)

\bibliographystyle{plain}
\bibliography{main}

\end{document}

