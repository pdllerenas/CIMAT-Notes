\documentclass{article}
\usepackage[yyyymmdd]{datetime}
\usepackage[nottoc]{tocbibind}
\usepackage{url}
\usepackage{array}
\usepackage[inline,shortlabels]{enumitem}
\usepackage{hhline}
\usepackage{multirow}
\usepackage{geometry}
\usepackage[dvipsnames]{xcolor}
\usepackage{colortbl}
\usepackage[framemethod=TikZ]{mdframed}
\usepackage{amsfonts}
\usepackage{amsmath}
\usepackage{tikz}
\usepackage{amsthm}
\newtheoremstyle{problemstyle}{3pt}{3pt}{\normalfont}{}{\bfseries}{\normalfont\bfseries:}{.5em}{}
\theoremstyle{problemstyle}
\newmdtheoremenv[
  linewidth=1pt,
  linecolor=RoyalBlue,
  backgroundcolor=RoyalBlue!10,
  roundcorner=5pt,
  innertopmargin=6pt,
  innerbottommargin=6pt,
  innerleftmargin=6pt,
  innerrightmargin=6pt,
  nobreak=true
]{problem}{Problem}

% Example
\newmdtheoremenv[
  linewidth=1pt,
  linecolor=ForestGreen,
  backgroundcolor=ForestGreen!10,
  roundcorner=5pt,
  nobreak=true
]{example}{Example}

% Theorem
\newmdtheoremenv[
  linewidth=1pt,
  linecolor=BrickRed,
  backgroundcolor=BrickRed!10,
  roundcorner=5pt,
  nobreak=true
]{theorem}{Theorem}

% Remark
\newmdtheoremenv[
  linewidth=1pt,
  linecolor=Goldenrod,
  backgroundcolor=Goldenrod!10,
  roundcorner=5pt,
  nobreak=true
]{remark}{Remark}

% Solution
\newenvironment{solution}{%
  \begin{mdframed}[linewidth=0.8pt,linecolor=Gray,backgroundcolor=Gray!5,roundcorner=5pt]%
  \noindent\textbf{Solution.}%
}{%
\hfill $ \diamond $ 
  \end{mdframed}%
}

\usepackage{listings}


% Make header with name and date etc.
\usepackage{fancyhdr}
\lhead{Pedro D. Llerenas\\M\'etodos Num\'ericos I}
\rhead{\today\\Tarea II}
\thispagestyle{fancy}

\usepackage[utf8]{inputenc}
\setlength{\parindent}{0pt} % Don't indent new paragraphs
\setlength{\headheight}{24pt} 

\newcommand{\Z}{\mathbb Z}
\newcommand{\Q}{\mathbb Q}
\newcommand{\R}{\mathbb R}
\newcommand{\C}{\mathbb C}
\newcommand{\N}{\mathbb N}


\begin{document}

\section*{M\'etodos iterativos para ecuaciones no lineales}\label{chap:m_etodos_iterativos_para_ecuaciones_no_lineales} % (fold)

% chapter M\'etodos iterativos para ecuaciones no lineales (end)
\begin{problem}
Escribe un programa para calcular la constante matem\'atica $ e $, considerando la definici\'on
\[
	e = \lim_{n\to\infty} \left(1+\frac{1}{n}\right)^{n},
\]
es decir, calcula $ (1+1/n)^n $ para $ n = 10^k $, $ k = 1, 2,\dots, 20. $ Determina el error relativo y absoluto de las aproximaciones compar\'andolas con $ \exp(1) $. \textbf{(1 punto)}
\end{problem}
\begin{solution}
	\lstinputlisting[caption=Calcula el n-\'esimo t\'ermino del l\'imite de e, language=C, breaklines=true]{./Code/p1.c}

	Este nos genera la list
	\begin{align*}
    e_{10^{1}} &= 2.593742460100002\\
    e_{10^{2}} &= 2.704813829421528\\
    e_{10^{3}} &= 2.716923932235594\\
    e_{10^{4}} &= 2.718145926824926\\
    e_{10^{5}} &= 2.718268237192297\\
    e_{10^{6}} &= 2.718280469095753\\
    e_{10^{7}} &= 2.718281694132082\\
    e_{10^{8}} &= 2.718281798347358\\
    e_{10^{9}} &= 2.718282052011560\\
    e_{10^{10}} &= 2.718282053234788\\
    e_{10^{11}} &= 2.718282053357110\\
    e_{10^{12}} &= 2.718523496037238\\
    e_{10^{13}} &= 2.716110034086901\\
    e_{10^{14}} &= 2.716110034087023\\
    e_{10^{15}} &= 3.035035206549262\\
    e_{10^{16}} &= 1.000000000000000\\
    e_{10^{17}} &= 1.000000000000000\\
    e_{10^{18}} &= 1.000000000000000\\
    e_{10^{19}} &= 1.000000000000000\\
    e_{10^{20}} &= 1.000000000000000
	\end{align*}
  Notemos que en el t\'ermino 15, la computadora deja de producir un valor prudente. Esto se debe a que un doble tiene precision de 15 digitos. Para este punto, $1/10^{15}$ tiene el decimal en el \'ultimo punto de precisi\'on. Entonces, al elevar a este mismo t\'ermino, quedamos con algo que ser\'a impreciso. Para el t\'ermino 16, $1/10^{16}$ ya es redondeado a 0 por el sistema. Entonces, solamente nos queda $ 1^{10^k} = 1 $.
\end{solution}

\begin{problem}
La ecuaci\'on $ x^3+x = 6 $ tiene una ra\'iz en el intervalo $ [1.55, 1.75] $, ¿cu\'antas iteracions se necesitan para obtener una aproximaci\'on de la raiz con error menor a 0.0001 con el m\'etodo de bisecci\'on? Verifica con el m\'etodo de bisecci\'on tu predicci\'on de la ra\'iz. \textbf{(2 puntos)}
\end{problem}
\begin{solution}
  
\end{solution}

\begin{problem}
Hallar una ra\'iz de $ f(x) = x^4 + 3x^2 - 2 $ por medio de las siguientes 4 formulaciones de punto fijo utilizando $ p_0 = 1 $:
\begin{center}
	\begin{enumerate*}[label=\alph*),itemjoin=\qquad]

		\item $\displaystyle x = \sqrt{\frac{2-x^4}{3}} $,
		\item $\displaystyle x = (2-3x^2)^{\frac{1}{4}}$,
		\item $\displaystyle x = \frac{2-x^4}{3x} $,
		\item $\displaystyle x = \left(\frac{2-3x^2}{x}\right)^{\frac{1}{3}}$
	\end{enumerate*}
\end{center}

\begin{enumerate}
	\item Las ra\'ices de $ f(x) $ deben de coincidir con las ra\'ices de $ x-g(x) $. Grafica $ f(x) $ y $ x-g(x) $. Comenta lo observado. \textbf{(1 punto)}

	\item Crea una talba comparativa para comparar el resultado de las raices de $f(x) $ con la raiz alcanzada con cada una de las formulaciones. Usa maximo 20 iteraciones y tol = 0.0001. Explica lo sucedido. \textbf{(2 puntos)}
\end{enumerate}
\end{problem}
\begin{solution}

\end{solution}

\begin{problem}
Utiliza el m\'etodo de bisecci\'on, m\'etodo de Newton, m\'etodo de la secante y m\'etodo de la falsa posici\'on para comparar los resultados de los siguientes problemas:
Encontrar $ \lambda $ con una presici\'on de $ 10^{-4} $ y $ N_{iter, max} = 100 $, para a ecuaci\'on de la poblaci\'on en t\'erminos de la tasa de natalidad $ \lambda $,
\[
	P(\lambda) = 1,000,000 e^{\lambda} + \frac{435,000}{\lambda}(e^{\lambda} - 1)
\]
para $ P(\lambda) = 1,564,000 $ individuos por a\~nos. Usa $ \lambda_0 = 0.01 $. (Sugerencia: graficar $ P(\lambda) - N $) \textbf{(4 puntos)}
\end{problem}


\bibliographystyle{plain}
\bibliography{references}

\end{document}
