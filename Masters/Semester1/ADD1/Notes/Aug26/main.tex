\documentclass{article}
\usepackage[yyyymmdd]{datetime}
\usepackage[shortlabels]{enumitem}
\usepackage[nottoc]{tocbibind}
\usepackage{url}
\usepackage{geometry}
\usepackage{xcolor}
\usepackage{mdframed}
\usepackage{float}
\usepackage{amsfonts}
\usepackage{amssymb}
\usepackage{mathtools}
\usepackage{amsthm}
\newtheorem{theorem}{Theorem}
\newtheorem{definition}{Definition}


\definecolor{pearl}{RGB}{234,224,220}
\definecolor{bg-se}{RGB}{246, 246, 246}
\usepackage{listings}
\usepackage{tikz}

\definecolor{clr-background}{RGB}{255,255,255}
\definecolor{clr-text}{RGB}{0,0,0}
\definecolor{clr-string}{RGB}{163,21,21}
\definecolor{clr-namespace}{RGB}{0,0,0}
\definecolor{clr-preprocessor}{RGB}{128,128,128}
\definecolor{clr-keyword}{RGB}{0,0,255}
\definecolor{clr-type}{RGB}{43,145,175}
\definecolor{clr-variable}{RGB}{0,0,0}
\definecolor{clr-constant}{RGB}{111,0,138} % macro color
\definecolor{clr-comment}{RGB}{0,128,0}

\lstdefinestyle{VS2017}{
	backgroundcolor=\color{pearl},
	basicstyle=\color{clr-text}, % any text
	stringstyle=\color{clr-string},
	identifierstyle=\color{clr-variable}, % just about anything that isn't a directive, comment, string or known type
	commentstyle=\color{clr-comment},
	directivestyle=\color{clr-preprocessor}, % preprocessor commands
	% listings doesn't differentiate between types and keywords (e.g. int vs return)
	breakatwhitespace=false,         
	breaklines=true, 
	% use the user types color
	keywordstyle=\color{clr-type},
	keywordstyle={[2]\color{clr-constant}}, % you'll need to define these or use a custom language
	tabsize=2
}

% Make header with name and date etc.
\usepackage{fancyhdr}
\lhead{Pedro D. Llerenas\\An\'alisis de datos}
\rhead{\today}
\thispagestyle{fancy}

\usepackage[utf8]{inputenc}
\setlength{\parindent}{0pt} % Don't indent new paragraphs
\setlength{\headheight}{24pt} 

\newcommand{\Z}{{\mathbb Z}}
\newcommand{\Q}{\mathbb Q}
\newcommand{\R}{\mathbb R}
\newcommand{\C}{\mathbb C}


\begin{document}
	
\section{Distribuciones condicionales}\label{sec:distribuciones_condicionales} % (fold)
 Before, we wrote
 \[
  P(X\in S)
 \]
 Now, we will write 
 \[
   P(X\in S|A) \coloneqq \frac{P(\{w:X(w)\in S\}\cap A)}{P(A)}
 \]

En general, 
\[
  X=(X_1, \dots, X_d)
\]

\[
  P(X_i\in A | X_j \in B) = \displaystyle\frac{P(X_i)\in A, X_j \in B}{P(X_j \in B)}
\]

\subsection{Independence of random variable}

X, Y are independent random variables random variables if 
\begin{align*}
  \forall A,B: P(x\in A|y\in B) = P(x\in A)\\
  \forall A,B: P(y\in B|x\in A) = P(y\in B)\\
  \forall A,B: P(x\in A,y\in B) = P(x\in A)P(y\in B)
\end{align*}

% section Distribuciones condicionales (end)	

\begin{definition}
  We say that 
  \begin{displaymath}
    X \sim Y
  \end{displaymath}
  
  if the distribution of $ X $ equals the distribution of $ Y $.
\end{definition}

\subsection{Recap}

\begin{definition}
  Let $ X $ be a discrete random variable. We define de \textit{expected value} of $ X $ as
  \begin{displaymath}
    E[X] = \sum x \cdot P(X=x)
  \end{displaymath}
  
\end{definition}
We note that
\[
  \sum (x-E[X])\cdot P(X=x) = \sum x\cdot P(X = x) - E[X]\sum P(X=x) = 0
\]

\begin{definition}
  For any $ g:X \to Y$,
  \begin{displaymath}
    E[g(X)] = \sum g(x)P(x=X)
  \end{displaymath}
  
\end{definition}

\begin{theorem}
  $ E $ is linear.
  \begin{displaymath}
    E(\alpha X_1 + \beta X_2) = \alpha E[X_1] + \beta E[X_2]
  \end{displaymath}
  
\end{theorem}
\begin{proof}
  
  \begin{align*}
    E(\alpha X_1 + \beta X_2) &= \sum_{(x_1, x_2)} (\alpha x_1 + \beta x_2)P(X=(x_1,x_2))\\
                              &= \sum_{x_1}\sum_{x_2} (\alpha x_1 +\beta x_2)P(x_1=X_1, x_2=X_2)\\
                              &= \sum_{x_1}\sum_{x_2}\alpha x_1P(x_1=X_1, x_2=X_2) + \sum_{x_1}\sum_{x_2} \beta x_2P(x_1=X_1, x_2=X_2)\\
                              &= \sum_{x_1} \alpha x_1 P(x_1=X_1) + \sum_{x_2}\beta x_2P(x_2=X_2)
  \end{align*}
  
\end{proof}



	
\end{document}

