\documentclass{article}
\newcommand\numberthis{\addtocounter{equation}{1}\tag{\theequation}}
\def\MinimumPaperHeight{210mm}
\usepackage{geometry}
\geometry{paperwidth=\MinimumPaperHeight,paperheight=\maxdimen,margin=1in}
\usepackage{etoolbox}
\usepackage{environ}
\NewEnviron{MyBox}{
    \setbox0=\vbox{\BODY}%
}{%
    \dimen0=\dp0%
    \pdfpageheight=\dimexpr\ht0+2cm\relax%
    \ifdim\pdfpageheight<\MinimumPaperHeight \pdfpageheight=\MinimumPaperHeight \fi%
    \unvbox0\kern-\dimen0%
}
\usepackage{etoolbox}

\AtBeginDocument{
  \setbox0=\vbox\bgroup
  \preto\enddocument{\egroup
    \dimen0=\dp0
    \pdfpageheight=\dimexpr\ht0+3.5cm\relax
    \ifdim\pdfpageheight<\MinimumPaperHeight
      \pdfpageheight=\MinimumPaperHeight
    \fi
    \unvbox0\kern-\dimen0 }
}
\usepackage[yyyymmdd]{datetime}
\usepackage[inline,shortlabels]{enumitem}
\usepackage{mathtools}
\usepackage[dvipsnames]{xcolor}
\usepackage[framemethod=TikZ]{mdframed}
\usepackage{amsfonts}
\usepackage{amsthm}
\usepackage{amsmath}
\newtheoremstyle{problemstyle}{3pt}{3pt}{\normalfont}{}{\bfseries}{\normalfont\bfseries:}{.5em}{}
\theoremstyle{problemstyle}
\newmdtheoremenv[
  linewidth=1pt,
  linecolor=RoyalBlue,
  backgroundcolor=RoyalBlue!10,
  roundcorner=5pt,
  innertopmargin=6pt,
  innerbottommargin=6pt,
  innerleftmargin=6pt,
  innerrightmargin=6pt,
  nobreak=true
]{problem}{Problem}

% Example
\newmdtheoremenv[
  linewidth=1pt,
  linecolor=ForestGreen,
  backgroundcolor=ForestGreen!10,
  roundcorner=5pt,
  nobreak=true
]{example}{Example}

% Theorem
\newmdtheoremenv[
  linewidth=1pt,
  linecolor=BrickRed,
  backgroundcolor=BrickRed!10,
  roundcorner=5pt,
  nobreak=true
]{theorem}{Theorem}

% Remark
\newmdtheoremenv[
  linewidth=1pt,
  linecolor=Goldenrod,
  backgroundcolor=Goldenrod!10,
  roundcorner=5pt,
  nobreak=true
]{remark}{Remark}

% Solution
\newenvironment{solution}{%
  \begin{mdframed}[linewidth=0.8pt,linecolor=Gray,backgroundcolor=Gray!5,roundcorner=5pt]%
  \noindent\textbf{Solution.}%
}{%
\hfill $ \qed $ 
  \end{mdframed}%
}

% Make header with name and date etc.
\usepackage{fancyhdr}
\lhead{Pedro D. Llerenas\\An\'alisis de Datos I}
\rhead{\today\\Tarea IV}
\thispagestyle{fancy}

\usepackage[utf8]{inputenc}
\setlength{\parindent}{0pt} % Don't indent new paragraphs
\setlength{\headheight}{24pt} 

\newcommand{\Z}{\mathbb Z}
\newcommand{\Q}{\mathbb Q}
\newcommand{\R}{\mathbb R}
\newcommand{\C}{\mathbb C}
\newcommand{\N}{\mathbb N}
\newcommand{\abs}[1]{\lvert #1 \rvert}
\DeclareMathOperator{\Var}{\mathbf{Var}}
\DeclareMathOperator{\E}{\mathbf{E}}

\begin{document}
%begin problem 1
\begin{problem}
Lee el siguiente articulo que se public\'o en septiembre de 2022 en El
Universal, a ra\'iz de que tanto en 1985, 2017 y 2022 hubo un sismo en la CDMX
el dia 19 de septiembre (adem\'as de que el 19 de septiembre 2022 hubo un
seminario de CC en CIMAT sobre m\'etodos num\'ericos para datos
sismol\'ogicos).
¿Cu\'al es tu opini\'on basada en lo que vimos en (apenas) un mes de clases de
probabilidad?
\end{problem}
%end problem 1

%begin solution 1
\begin{solution}

\end{solution}
%end solution 1

%begin problem 2
\begin{problem}
Sea $ X $ el tiempo en minutos entre la llegada de un cliente a una oficina de
correos y la finalizaci\'on de la atenci\'on, y $ Y $ el tiempo que el cliente
tiene que esperar. Se sabe que la densidad conjunta es:
\begin{align*}
	f_{X,Y}(x,y) =
	\begin{cases}
		ky\exp(-x) & \text{ if } 0\leq y<x<\infty \\
		0          & \text{ otherwise}.
	\end{cases}
\end{align*}
Determina el valor de $ k $, $ Var(X|Y=1) $ y $ P(X-Y<a) $ con $ a>0 $.
\end{problem}
%end problem 2

%begin solution 2
\begin{solution}
	Para encontrar $ k $, integramos sobre todo $ \R^2 $. Esta integral debe tener valor $ 1 $; es decir,
	\begin{align*}
		\displaystyle\int_{\R^2} f_{X,Y}(x,y)\,dA = 1.
	\end{align*}
	Por la definici\'on de $ f_{X,Y}(x,y) $ y linealidad de la integral, podemos descomponerla en la regi\'on $ 0\leq y < x<\infty $ y su complemento. Como en el complemento es 0, tenemos
	\begin{align*}
		1=\displaystyle\int_{\R^2} f_{X,Y}(x,y)\,dA & = \int_{0}^{\infty}\int_{0}^{x} f_{X,Y}(x,y)\,dydx                                                     \\
		                                            & = \int_{0}^{\infty}\int_{0}^{x} kye^{-x}\,dydx                                                         \\
		                                            & = \frac{k}{2}\int_{0}^{\infty}x^2e^{-x}\,dx                                                            \\
		(\text{por partes}\times 2)                 & = \frac{k}{2}\left[-x^2e^{-x} - 2xe^{-x}\right]_{0}^{\infty} + \frac{k}{2}\int_{0}^{\infty}2e^{-x}\,dx \\
		                                            & = 0 - k [e^{-x}]_{0}^{\infty}                                                                          \\
		                                            & = k.
	\end{align*}
	Es decir, $ k=1 $ y
	\begin{align*}
		f_{X,Y}(x,y) =
		\begin{cases}
			y\exp(-x) & \text{ if } 0\leq y<x<\infty \\
			0         & \text{ otherwise}.
		\end{cases}
	\end{align*}

	Recordemos que
	\begin{align*}
		\Var(X) \coloneqq \int(x-\E[X])^2f_{X}(x)\,dx,
	\end{align*}
	por lo que tenemos
	\begin{align*}
		\Var(X|Y=1) \coloneqq \int(x-\E[X|Y=1])^2f_{X|Y=1}(x)\,dx.
	\end{align*}
	Entonces, calculemos $ \E[X|Y=1] $. Recordemos que
	\begin{align*}
		f_{X|Y=y_0}(x) = \frac{f_{X,Y}(x,y_0)}{f_Y(y_0)},
	\end{align*}
	por lo que, en este caso, como $ f_Y(y) = y $,
	\begin{align*}
		f_{X|Y=1}(x) & = \frac{f_{X,Y}(x,1)}{f_Y(1)}          \\
		             & = \frac{f_{X,Y}(x,1)}{1}               \\
		             & =\begin{cases}
			                e^{-x}, & \text{ if } x\in [0,\infty] \\
			                0,      & \text{ otherwise}
		                \end{cases}
	\end{align*}
	Se sigue que
	\begin{align*}
		\E[X|Y=1]           & = \int_{\R} x f_{X|Y=1}(x)\,dx                    \\
		                    & = \int_{[0,\infty]} x e^{-x}\,dx                  \\
		(\text{por partes}) & = [xe^{-x}]_{0}^{\infty}+\int_{0}^{\infty} e^{-x} \\
		                    & = 0 + 1                                           \\
		                    & = 1.
	\end{align*}
	Volviendo a $ \Var(X|Y=1) $, tenemos entonces
	\begin{align*}
		\Var(X|Y=1) & = \int_{\R} (x-1)^2 f_{X|Y=1}(x)\,dx                                                                \\
		            & = \int_{0}^{\infty} (x-1)^2 e^{-x}\,dx                                                              \\
		            & = \int_{0}^{\infty} x^2 e^{-x}\,dx - \int_{0}^{\infty} 2x e^{-x}\,dx + \int_{0}^{\infty} e^{-x}\,dx \\
		            & = 2 - 2 + 1\numberthis\label{eq:1}                                                                  \\
		            & = 1.
	\end{align*}
	Notemos que \eqref{eq:1} viene de los c\'alculos de integrales anteriores.

	Finalmente, calculamos $ P(X-Y<a) $. Dividimos esto en dos casos: $ x > a $ y $ x < a $. Primero $ x > a $:
	\begin{align*}
		P(X-Y<a) & = \int\int_{x-y<a} f_{X,Y}(x,y)\,dx\,dy                                                   \\
		         & =\int\int_{x-y<a}ye^{-x}\,dx\,dy                                                          \\
		         & =\int_{0}^{\infty}\int_{x-a}^{x}ye^{-x}\,dy\,dx                                           \\
		         & =\int_{0}^{\infty}\frac{(x^{2} - (x-a)^{2})}{2}e^{-x}\,dx                                 \\
		         & =\frac{1}{2}\int_{0}^{\infty}(2xa - a^2)e^{-x}\,dx                                        \\
		         & =\frac{1}{2}\left(\int_{0}^{\infty}2xae^{-x}\,dx - \int_{0}^{\infty}a^2 e^{-x}\,dx\right) \\
		         & =\frac{1}{2}\left(2a - a^2\right)                                                         \\
		         & =\frac{a(2-a)}{2}.
	\end{align*}
	En el otro caso, si $ x < a $, tenemos
	\begin{align*}
		P(X-Y<a) & = \int\int_{x-y<a} f_{X,Y}(x,y)\,dx\,dy       \\
		         & =\int\int_{x-y<a}ye^{-x}\,dx\,dy              \\
		         & =\int_{0}^{\infty}\int_{0}^{x}ye^{-x}\,dy\,dx \\
		         & =\int_{0}^{\infty}\frac{x^{2}}{2}e^{-x}\,dx   \\
		         & = \frac{1}{2}.
	\end{align*}
\end{solution}
%end solution 2

\begin{problem}
Al mismo tiempo alguien sale de Valenciana a Dos Rios y otra persona sale de
Dos Rios a Valenciana con velocidad constante. La velocidad de la primera
persona es un valor de una distribuci\'on $ \mathcal U([20, 40])$ km/h y la
velocidad de la segunda persona es siempre 45km/h. Calcula el tiempo promedio
que tardan en encontrarse si sabes que la distancia entre los dos lugares es
3.5km.
\end{problem}
\begin{solution}
	Sean $ V_1, V_2 $ las velocidades de la persona en Valenciana y la persona en Dos Rios, respectivamente. Supongamos que Valenciana se encuentra en el origen, y Dos Rios a $ x_0 = 3.5 $km del origen. Entonces, el tiempo $ t $ que buscamos es uno tal que
	\begin{align*}
		v_1 t = -v_2 t + x_0\quad \iff \quad t = \frac{x_0}{v_1+v_2}.
	\end{align*}
	Por lo que, el valor esperado del tiempo se calcula de la siguiente manera:
	\begin{align*}
		\E[T] & = \E\left[\frac{x_0}{V_1+V_2}\right]                    \\
		      & = \int_{\R} \frac{x_0}{v_1+v_2} f_{V_1}(v_1)\,dv_1      \\
		      & = \frac{1}{20}\int_{20}^{40} \frac{x_0}{v_1+v_2} \,dv_1 \\
		      & = \frac{x_0}{20} \log\frac{\abs{40+v_2}}{\abs{20+v_2}}  \\
		      & = \frac{3.5}{20}\log \frac{17}{13}                      \\
		      & \approx  0.175\cdot 0.268                               \\
		      & \approx 0.046
	\end{align*}
	Esto es, $ 0.046 $ horas, que equivale a aproximadamente $ 169 $ segundos.
\end{solution}

\begin{problem}
La vida \'util de un cierto tablet sigue una distribuci\'on normal con promedio
3 a\~nos y $ \sigma^2 = 0.9$.
\begin{enumerate}[a)]
	\item Calcula la probabilidad que funcionar\'a m\'as de cuatro a\~nos.
	\item Se quiere determinar la duraci\'on de la garant\'ia. ¿Para cu\'antos
	      meses m\'aximos se debe dar la garant\'ia para que la probabilidad que el
	      tablet se descomponga antes del fin de la garant\'ia sea no mayor que 0.25?
	      La respuesta debe ser en t\'erminos de meses enteros.
\end{enumerate}
\end{problem}
\begin{solution}
	\begin{enumerate}[a)]
		\item La distribuci\'on del tiempo de vida \'util $ T\sim \mathcal N(3, 0.9) $. Queremos $ P(T>4) $. Primero, normalizamos la distribuci\'on:
		      \begin{align*}
			      Z = (T-3)/0.9\sim \mathcal N(0,1).
		      \end{align*}
		      Entonces, $ P(T>4) = P(0.9Z + 3 > 4) = P(Z > 10/9)\approx 0.1333 $. Esta \'ultima aproximaci\'on viene de una calculadora.
		\item Buscamos el n\'umero m\'aximo de meses $ M $ tal que $ P(T<M)\leq
			      0.25 $. Es decir, $ P(T<M+1) >0.25 $. Nuevamente, usando la
		      distribuci\'on normalizada, tenemos
		      \begin{align*}
            P(T<M) = P(0.9Z+3<M) = P\left(Z < \frac{M-3}{0.9}\right) \leq 0.25
		      \end{align*}
          Notemos que para $ M=\frac{28}{12} $, tenemos $ P(Z<-0.74) \approx 0.23 $, pero para $ M = \frac{29}{12} $ tenemos $ P(Z<-0.64)\approx 0.26 $. Entonces, debemos dar una garant\'ia de 28 meses.
	\end{enumerate}
\end{solution}
\end{document}
