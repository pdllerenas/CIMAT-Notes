\documentclass{article}
\usepackage[yyyymmdd]{datetime}
\usepackage[shortlabels]{enumitem}
\usepackage[nottoc]{tocbibind}
\usepackage{url}
\usepackage{array}
\usepackage{hhline}
\usepackage{multirow}
\usepackage{geometry}
\usepackage[dvipsnames]{xcolor}
\usepackage{colortbl}
\usepackage[framemethod=TikZ]{mdframed}
\usepackage{amsfonts}
\usepackage{tikz}
\usepackage{amsthm}
\newtheoremstyle{problemstyle}{3pt}{3pt}{\normalfont}{}{\bfseries}{\normalfont\bfseries:}{.5em}{}
\theoremstyle{problemstyle}
\newmdtheoremenv[
  linewidth=1pt,
  linecolor=RoyalBlue,
  backgroundcolor=RoyalBlue!10,
  roundcorner=5pt,
  innertopmargin=6pt,
  innerbottommargin=6pt,
  innerleftmargin=6pt,
  innerrightmargin=6pt,
  nobreak=true
]{problem}{Problem}

% Example
\newmdtheoremenv[
  linewidth=1pt,
  linecolor=ForestGreen,
  backgroundcolor=ForestGreen!10,
  roundcorner=5pt,
  nobreak=true
]{example}{Example}

% Theorem
\newmdtheoremenv[
  linewidth=1pt,
  linecolor=BrickRed,
  backgroundcolor=BrickRed!10,
  roundcorner=5pt,
  nobreak=true
]{theorem}{Theorem}

% Remark
\newmdtheoremenv[
  linewidth=1pt,
  linecolor=Goldenrod,
  backgroundcolor=Goldenrod!10,
  roundcorner=5pt,
  nobreak=true
]{remark}{Remark}

% Solution
\newenvironment{solution}{%
  \begin{mdframed}[linewidth=0.8pt,linecolor=Gray,backgroundcolor=Gray!5,roundcorner=5pt, nobreak=true]%
  \noindent\textbf{Solution.}%
}{%
\hfill $ \diamond $ 
  \end{mdframed}%
}

\usepackage{listings}


% Make header with name and date etc.
\usepackage{fancyhdr}
\lhead{Pedro D. Llerenas\\Programaci\'on y Algoritmos I}
\rhead{\today\\Tarea II}
\thispagestyle{fancy}

\usepackage[utf8]{inputenc}
\setlength{\parindent}{0pt} % Don't indent new paragraphs
\setlength{\headheight}{24pt} 

\newcommand{\Z}{\mathbb Z}
\newcommand{\Q}{\mathbb Q}
\newcommand{\R}{\mathbb R}
\newcommand{\C}{\mathbb C}
\newcommand{\N}{\mathbb N}


\begin{document}
\section*{Strings}\label{sec:strings} % (fold)

\begin{problem}
(20\%) Escriba un programa que realice las siguientes operaciones sobre cadenas de caracteres (strings); no usar memoria din\'amica:
\begin{enumerate}[a)]
	\item \textbf{int longitud\_str(str)}: Funci\'on que regrese la longitud de un string \texttt{str} (sin contar el caracter nulo);
	\item \textbf{char *copia\_str(dst, src)}: copia el string \texttt{src} a \texttt{dst}, incluyendo el caracter nulo $\backslash$\texttt{0}; regresa un apuntador a \texttt{dst}.
	      \textbf{Nota:} La funci\'on debe verificar que los strings no est\'en superpuestos y si hay suficiente espacio en \texttt{dst} para copiar \texttt{src}; en caso que no, incluir solo hasta donde sea posible almacenar en \texttt{dst};
	\item \textbf{int compara\_str(str1, str2)}: Funci\'on que concatene tres strings (en el orden dado) separados por un espacio y los almacene en \texttt{str1}; debe regresar la nueva longitud de \texttt{str1};
	\item \textbf{int concatena\_str(str1, str2, str3)}: Funci\'on que concatene tres strings (en el orden dado) separados por un espacio y alamacene en \texttt{str1}; debe regresar la nueva longitud de \texttt{str1};
	      \textbf{Nota}: La funci\'on debe verificar que los strings no est\'en superpuestos y si hay suficiente espacio en \texttt{str1} para almacenar todos los strings; en caso que no, incluir solo hasta donde sea posible almacenar en \texttt{str1};
	\item \textbf{int encuentra\_str(str1, str2)}: Funci\'on que busque un sub-string \texttt{str1} en \texttt{str2} y regrese el n\'umero de veces que lo encuentra.
	\item \textbf{int **frecuencia(str)}: Encuentre la frecuencia de cada elemento de \texttt{str}, y regrese un arreglo bidimensional donde se almacene la letra (1er columna) y frecuencia (2da columna).
	      \textbf{Salida}: w --- 1, r --- 5, s --- 4, etc\dots
	      \textbf{Nota}: En este ejercicio puede usar memora est\'atica o din\'amica.

	\item \textbf{char *sin\_repetir(str)}: Funci\'on que encuentre las palabras en \texttt{str} que no tengan letras repetidas y las imprima (dentro de la misma funci\'on por simplicidad).
\end{enumerate}
\end{problem}
\begin{solution}

\end{solution}
% section Strings (end)


\section*{Arreglos Bidimensionales}\label{sec:arreglos_bidimensionales} % (fold)
\begin{problem}
  (10\%) Escribir un programa que genere una caminata aleatoria en una matriz de 10x10. El arreglo debe contener inicialmente puntos `.', y debe recorrerse basado en el residuo de un n\'umero aleatoria (usar \textbf{srand()} y \textbf{rand()}) cuyos resultado puede ser {0 (arriba), 1 (abajo), 2 (izq), 3 (der)}, que indican la direcci\'on a moverse. A) Verificar que el movimiento no se salga del arreglo de la matriz, y B) No se puede visitar el mismo lugar más de una vez. Si alguna de estas condiciones se presenta, intentar moverse hacia otra direcci\'on definida; si todas las posiciones están ocupadas, finalizar el programa e imprimir el resultado.
\end{problem}

\begin{solution}

\end{solution}

\begin{problem}
(10\%)
\begin{enumerate}[a)]
	\item Dado un arreglo bidimensional de enteros $ M\times N $, encontrar el m\'aximo valor para cada columna y cada rengl\'on:
	      \setlength\doublerulesep{0.5pt}\doublerulesepcolor{black}
	      \begin{center}
		      \begin{tabular}[c]{|c||c|c|c|c|}
			      \hline
			        & 9 & 6 & 8 & 1 \\
			      \hhline{|=#=|=|=|=|}
			      8 & 3 & 6 & 8 & 1 \\
			      \hline
			      4 & 2 & 4 & 3 & 1 \\
			      \hline
			      9 & 9 & 5 & 2 & 1 \\
			      \hline
		      \end{tabular}
	      \end{center}
	      \textbf{Nota}: Recorre el arreglo (columnas y renglones) en forma eficiente.

	\item Cuente el n\'umero de bytes del arreglo bidimensional con valor 0 (recuerde que cada entero est\'a representado por 4 bytes).
\end{enumerate}

\end{problem}
\begin{solution}

\end{solution}
% section Arreglos Bidimensionales (end)


\section*{Memoria Din\'amica}\label{sec:memoria_din_amica} % (fold)
\begin{problem}
(20\%) Dado una lista de nombres (string) de $ N $ personas (apellido\_paterno, apellido\_materno, nombre(s)), escribir una funci\'on que ordene los nombres alfab\'eticamente usando un arreglo de apuntadores:
\begin{center}
  char **crea\_arreglo(char **arr, \dots)\\
  char **ordena(char **arr,\dots)
\end{center}
Los nombres pueden tener distinta longitud, pero la memoria que ocupan debe ser la justa (sin desperdicio); cuando un nombre sea prefijo de otro, considerar al nombre m\'as corto como menor. El ordenamiento debe ser a trav\'es de una funci\'on que reciba el arreglo de apuntadores.
\end{problem}
\begin{solution}
  
\end{solution}

\begin{problem}
  (20\%) Separe un string en tokens de acuerdo a un caracter especial dado como entrada (puede ser espacio, /, \%, etc.) y que regrese un arreglo que apunte a cada uno de los tokens esperados:
\begin{center}
  char **tokens(char *str, char ch)
\end{center}
 o NULL en caso de no encontrar alg\'un token.
  
\end{problem}

\begin{solution}
  
\end{solution}

\begin{problem}
 Escriba una funci\'on que reciba $ N $ arreglos de enteros ordenados de menor a mayor, y mezcle los arreglos en un solo arreglo ordenado de igual forma. 

Prototipo de la funci\'on:
\textbf{int *merge(int **arr, int N, int *dim)}
donde \textbf{arr} tiene la siguiente estructura:

\textbf{N} es el n\'umero total de arreglos y \textbf{dim} es un arreglo de enteroscon la dimensi\'on de cada uno de los arreglos de entrada. La funci\'on regresa un apuntados hacia el arreglo mezclado y generado din\'amicamente dentro de la funci\'on \textbf{merge()}.

\textbf{Nota}: Generar din\'amicamente todos los arreglos necesarios, e inicialice cada arreglo con valores aleatorios mediante la funci\'on \textbf{rand()}.
\end{problem}
\begin{solution}
  
\end{solution}
% section Memoria Din\'amica (end)

\bibliographystyle{plain}
\bibliography{references}

\end{document}
